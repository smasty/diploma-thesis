%!TEX root = main.tex

\newpage
\chapter{Informačné bulletiny}

Informačné bulletiny sú aj v súčasnosti jedným z najrozšírenejších spôsobov, ako v~online
prostredí informovať používateľov o dianí na webovom portáli. Prevádzkovatelia webových portálov využívajú informačné
bulletiny na predstavenie nového obsahu, akciového tovaru, zaujímavostí z určitej oblasti alebo špeciálnych ponúk pre
svojich používateľov a zákazníkov. Informačné bulletiny sú tiež využívané ako prostriedok pre motiváciu používateľov
k opätovnej návšteve webového portálu.

Informačné bulletiny spravidla nadobúdajú formu e-mailu, ktorý je zvyčajne v~pravidelných intervaloch doručovaný
do schránok používateľov, ktorí o jeho doručovanie prejavili záujem.


\section{Systémy využívajúce informačné bulletiny}

Informačné bulletiny sú efektívnym spôsobom dosahu na používateľov vo viacerých druhoch webových portálov,
pričom pre každú kategóriu portálov spĺňajú mierne odlišné ciele, čomu sa prispôsobuje aj ich obsah:


\begin{my_itemize}
  \item{\textbf{Marketing~/~internetové obchody}\\
        Najčastejšie informačné bulletiny rozposielajú svojim používateľom
        práve internetové obchody, zľavové portály a iné marketingové weby. Tieto bulletiny zvyčajne ponúkajú používateľom
        zaujímavé produkty, rôzne zľavy alebo exkluzívne ponuky. Cieľom týchto bulletinov je získať pozornosť používateľa
        po tom, ako už úspešne na internetovom obchode nakúpil, aby tak urobil znova.}
  \item{\textbf{Tématické weby/blogy}\\
        Webové stránky alebo blogy, ktoré produkujú tématický obsah spravidla určený pre konkrétne záujmové skupiny používateľov,
        zvyčajne využívajú informačné bulletiny na informovanie používateľov o novom zaujímavom obsahu na stránke.
        Vzhľadom na kvantitu nového obsahu je zvyčajne prispôsobená aj frekvencia rozposielania bulletinov.}
  \item{\textbf{Komunitné portály}\\
        Komunitné portály sú odlišné hlavne tým, že väčšinu obsahu vytvárajú samotní používatelia. Informačné
        bulletiny týchto portálov spravidla obsahujú zoznam najnovších, najzaujímavejších, alebo najkontroverznejších
        príspevkov. Okrem toho môžu obsahovať aj informácie alebo správy od moderátorov a prevádzkovateľov portálu, prípadne
        obsah z iných pridružených webov, napr. blogov.}
  \item{\textbf{CQA systémy}\\
        CQA systémy patria medzi komunitné portály, preto aj ich informačné bulletiny zdieľajú podobný cieľ a štruktúru.
        Odlišujú sa však tým, že zvyčajne tieto systémy produkujú veľmi veľa obsahu. Pre používateľa preto môže byť
        problematické nájsť zaujímavý alebo relevantný obsah. Práve tento problém by mali riešť informačné bulletiny.}
\end{my_itemize}


Spôsob vytvárania informačných bulletinov nie je závislý len od druhu webového portálu, ale aj od množstva obsahu, ktorý
tento portál produkuje.

\textbf{Ručne zostavované bulletiny}\\
Zostavovanie informačných bulletinov ručne autorom, správcom či moderátorom webového portálu je únosné len v prípade, že
webový portál vyprodukuje za dané obdobie iba relatívne malé množstvo obsahu. Využívať sa môže hlavne v prípade tématických
blogov, kde úzke zameranie cieľovej skupiny používateľov zároveň umožňuje autorovi zostaviť pomerne relevantný informačný
bulletin, ktorý je pre používateľov zaujímavý a prínosný.

\textbf{Automaticky generované generické bulletiny}\\
Najčastejším druhom informačných bulletinov sú bulletiny, ktoré obsahujú len generický obsah, napr. zoznam najpredávanejších
produktov v internetovom obchode, alebo produkty s najväčšími zľavami. Takýto bulletin sa zostavuje pomerne jednoducho,
no jeho prínos je otázny vzhľadom na veľkú diverzitu obsahu aj používateľov.

\textbf{Automaticky generováné personalizované bulletiny}\\
Najefektívnejším spôsobom, ako zostaviť relevantný bulletin pre veľké množstvo používateľov alebo z veľkého množstva obsahu,
je využitie metód personalizácie a odporúčania napr. na základe predošlej aktivity používateľa. Takýto prístup umožňuje
zostaviť pre každého používateľa bulletin, ktorý má najväčšiu šancu splniť svoj cieľ -- či už je to nákup ďalších produktov,
alebo zvýšenie návštevnosti portálu.


\section{Problémy informačných bulletinov}
Hlavným problémom informačných bulletinov je stále sa znižujúca miera interakcie používateľov
s informačnými bulletinmi.

Štúdia spoločnosti Silverpop z roku 2012~\cite{mailmarketing} na vzorke informačných bulletinov 1124 spoločností ukázala,
že počet používateľov, ktorí vôbec otvoria informačný bulletin sa pohybuje na úrovni 20\% a stále klesá. Navyše konkrétne
v oblasti technológií sa táto hodnota pohybuje len na 16,5\%. Ešte menšia je miera preklikov
(angl. \emph{Click-through rate - CTR}), ktorá sa celkovo pohybuje na úrovni 5,4\% a~v~prípade technologicky zameraných
informačných bulletinov len 3,6\%. Napriek tomu sa miera odhlásení z odoberania (angl. \emph{unsubscribe rate}) pohybuje
len na úrovni 2\%.

Dôvodov, prečo používatelia prejavujú iba malý záujem o informačné bulletiny, ktoré im sú doručované, môže byť niekoľko.
Jedným z takýchto dôvodov môže byť vysoká saturácia -- používateľom chodí priveľké množstvo informačných bulletinov,
dôsledkom čoho používatelia rezignujú a tieto e-maily ani neotvárajú.
Hlavným nedostatkom informačných bulletinov, a zároveň dôvodom, prečo iba 5\% používateľov klikne na obsah v informačnom
bulletine, je však relevancia ponúkaného obsahu.


\section{Diskusia}

Množstvo webových portálov doručuje všetkým svojim používateľom presne ten istý obsah informačného bulletinu.
Často je tento obsah vytváraný manuálne editormi, a zameriava sa len všeobecne na aktuálne dianie na danom webovom
portáli. Takýto všeobecný informačný bulletin však nutne nemôže byť dostatočne relevantný pre značnú časť používateľov.

Riešením problému relevancie informačných bulletinov je vytváranie personalizovaných informačných bulletinov, ktoré
každému používateľovi ponúkajú len ten obsah, ktorý je pre neho najzaujímavejší a najrelevantnejší.

Kvalitné informačné bulletiny sú obzvlášť dôležité pre webové portály, ktoré obsahujú veľké množstvo diverzného obsahu.
Medzi takéto portály patria aj systémy pre odpovedanie na otázky -- CQA systémy, ktoré tvorí primárne používateľmi
vytváraný obsah. Pre tieto systémy nie je efektívne vytvárať informačné bulletiny ručne, ani poskytovať len generické
informačné bulletiny.


%%%%%%%%%%%%%%%%%%%%%%%%%%%%%%%%%%%%%%%%%%%%%%%%%%%%%%%%%%%%%%%%%%%%%%%%%%%%%%%%%%%%%%%%%%%%%%%%%%%%%%%%%%%%%%%%%%%%%%%%


\newpage
\chapter{CQA systémy}

CQA systémy sú jednou z výrazných skupín webových portálov, ktoré sú založené na princípe používateľsky vytváraného obsahu.
Tieto systémy umožňujú používateľom položiť otázky, ktoré nie je možné zodpovedať použitím štandardných vyhľadávačov~\cite{Liu2012}
a zároveň odpovedať na otázky iných používateľov.

Napriek tomu, že väčšina CQA systémov sa spočiatku zameriava najmä na poskytnutie zmysluplnej odpovede na konkrétnu otázku,
v súčasnosti je možné v prípade niektorých CQA systémov (napr. Stack Overflow) vnímať postupnú zmenu zamerania
z jednorázových odpovedí na kolaboratívne vytváranie komplexnejších poznatkov s dlhodobou hodnotou~\cite{Anderson2012}.
Za týmto účelom CQA systémy implementujú hlasovanie a princíp reputácie ako spôsob podpory komunitného aspektu označovania
najlepších odpovedí na položené otázky.


\section{Druhy CQA systémov}

CQA systémy možno kategorizovať do dvoch základných skupín podľa toho, na akú oblasť otázok sa tieto systémy zameriavajú.

\textbf{Univerzálne CQA systémy}\\
CQA systémy ako \emph{Yahoo! Answers}, \emph{Wiki Answers} alebo \emph{Quora} nie sú zamerané na konkrétne oblasti
a umožňujú používateľom pokladať otázky na akékoľvek témy~\cite{Chua2014}.

Tento druh CQA systémov má štandardne vyšší počet používateľov aj aktivity ako úzko špecializované CQA systémy,
no tiež tu existuje väčšia pravdepodobnosť výskytu nekvalitných, jednoduchých alebo neužitočných otázok a odpovedí,
ako aj veľký počet duplicitných otázok, ktoré už boli zodpovedané. Zároveň sú univerzálne CQA systémy zamerané viac
na samotný proces kladenia otázok a~odpovedania na ne, než na vytváranie dlhodobo hodnotného obsahu.

\textbf{Úzko špecializované CQA systémy}\\
Opakom univerzálnych CQA systémov sú CQA systémy, ktoré sú špecializované na konkrétne oblasti záujmu.
Medzi takéto CQA systémy patria napríklad jednotlivé komunity v rámci siete Stack Exchange, ktorá zahŕňa rôzne druhy
komunít, od všeobecnejších, ako je napr. komunita venujúca sa matematike\footnote{\url{https://math.stackexchange.com}},
po veľmi úzko špecializované, akými sú napr. komunity \emph{Ask Ubuntu}\footnote{\url{https://askubuntu.com}} alebo
\emph{Raspberry Pi}\footnote{\url{https://raspberrypi.stackexchange.com}} venujúce sa konkrétnym produktom.

Tématicky zamerané CQA systémy majú väčší potenciál pre vznik dlhodobo hodnotného obsahu~\cite{Anderson2012}. V rámci
týchto systémov tiež vzniká množstvo prepojení medzi obsahom \textit{(otázky podobného charakteru, riešenie problému
v príbuznej oblasti)}, čo vedie k vzniku \emph{znalostných sietí}~(angl.~\emph{knowledge networks})~\cite{Li2016}.
S cieľom zvýšiť hodnotu jednotlivých príspevkov tiež mnohé CQA systémy zavádzajú možnosť komunitnej úpravy
otázok a odpovedí~\cite{Li2015}, čo vedie okrem zvýšenej aktivity aj k zvýšeniu vnímanej užitočnosti príspevku.


\section{Problémy CQA systémov}

CQA systémy sa musia vysporiadavať s tými istými druhmi problémov, ako iné kategórie systémov založené na používateľmi
vytvorenom obsahu.

\subsection{Problém dlhého chvosta v aktivite používateľov}\label{cqa:tail}
Čím viac sa zvýrazňuje trend orientácie CQA systémov skôr na poskytovanie obsahu s dlhodobou hodnotou ako na samotné
poskytnutie odpovede na položenú otázku, tým viac sa prehlbuje problém \emph{dlhého chvosta} (angl. \emph{long tail}).
Ide o štandardný problém všetkých stránok zameriavajúcich sa na používateľmi vytváraný obsah, kedy je veľká väčšina
používateľov týchto stránok len pasívnymi čitateľmi (angl. \emph{lurkers}) a najväčšia časť obsahu je vytvorená len veľmi
úzkou skupinou najaktívnejších používateľov.

V prípade CQA systému Stack Overflow sa podiel aktívnych používateľov (takých, ktorí za sledovaný mesiac pridali
do systému aspoň jednu otázku alebo odpoveď) za marec 2017\footnote{Výsledky za aktuálne obdobie boli získané prostredníctvom nástroja Stack
Exchange Data Explorer -- \\\url{https://data.stackexchange.com}} pohyboval na úrovni 3\% všetkých
používateľov~\cite{Srba2016SOFail}.

\subsection{Variabilita v kvalite obsahu}
Ďalším problémom CQA systémov je variabilná kvalita otázok a odpovedí v týchto systémoch. Zvyšujúcou sa popularitou CQA
systémov narastá aj podiel obsahu s~nízkou kvalitou, či už vo forme veľmi jednoduchých otázok alebo nedostatočne
podrobných odpovedí, ako aj množstvo duplicitných otázok -- otázok, ktoré už boli v systéme
zodpovedané~\cite{Srba2016SOFail,Ponzanelli2014}.

Jedným z riešení tohto problému, ktorý využíva napr. CQA platforma Stack Exchange, je komunitné zabezpečovanie kvality obsahu
prostredníctvom moderátorov -- používateľov s oprávnením upravovať, označiť duplikáty alebo vymazať obsah.


\section{Informačné bulletiny v CQA systémoch}

Význam informačných bulletinov narastá v rámci systémov pre odpovedanie na otázky, ktoré sú prominentným druhom
online komunít produkujúcich veľké množstvo používateľmi vytváraného obsahu.

Súčasný výskum v oblasti CQA systémov~\cite{Srba2016} sa venuje predovšetkým oblastiam skúmania správania používateľov,
smerovania a odporúčania otázok a kvality otázok a odpovedí v týchto systémoch. Problematike vytvárania informačných
bulletinov v doméne CQA systémov zatiaľ nebola venovaná veľká pozornosť.

Mnohé populárne CQA systémy aj v súčasnosti ponúkajú svojim používateľom informačné bulletiny majúce iba generický
charakter a nijakým spôsobom neuvažujú relevantnosť obsahu pre konkrétnych používateľov, prípadne informačné bulletiny
neponúkajú vôbec.

\subsection{Informačné bulletiny v sieti Stack Exchange}\label{so-newsletter}

Sieť Stack Exchange\footnote{ \url{https://stackexchange.com}}, ktorá patrí medzi najpopulárnejšie CQA systémy súčasnosti,
sa skladá z viac ako 160 samostatných komunít zameraných na rôzne oblasti. Stack Exchange ponúka používateľom všetkých
komunít možnosť odoberať informačný bulletin, ktorý je doručovaný raz týždenne.

Informačné bulletiny komunít Stack Exchange obsahujú tri sekcie (Obr.~\ref{fig:so-newsletter}). Prvá sekcia je rovnaká
pre všetkých používateľov konkrétnej komunity a obsahuje zoznam najlepšie hodnotených nových otázok.
Obsah nasledujúcich dvoch sekcií je náhodne generovaný. Tieto sekcie obsahujú najpopulárnejšie otázky z predchádzajúceho
týždňa a náhodný výber nezodpovedaných otázok.

Používatelia CQA systému Stack Exchange nie sú s takýmto generickým informačným bulletinom
spokojní\footnote{\url{https://meta.stackexchange.com/q/247298}. Prevzaté 31.4.2017.}. Medzi problémy, ktoré najčastejšie
používatelia vytýkajú súčasnému informačnému bulletinu patria:

\begin{my_itemize}
  \item{\textbf{Náhodne generovaný obsah} -- Sekcia nezodpovedaných otázok obsahuje náhodný výber otázok bez odpovedí.
        Pravdepodobnosť, že používateľ vie na niektorú z nich odpovedať, je tak veľmi
        malá\footnote{\url{https://meta.stackexchange.com/q/96758}}.}
  \item{\textbf{Absencia personalizácie} -- Otázky v jednotlivých sekciách nijakým spôsobom nezohľadňujú používateľove
        obľúbené značky alebo jeho aktivitu. Dôsledkom hlavne pri väčších komunitách je tak nízka relevancia ponúkaného
        obsahu\footnote{\url{https://meta.stackexchange.com/q/110902}}.}
  \item{\textbf{Malá rôznorodosť obsahu} -- Hlavne pokročilejší a aktívnejší používatelia by chceli v informačnom bulletine vidieť
        okrem rôznych otázok aj iný obsah, okrem iného napr. rôzne štatistiky aktivity komunity, zoznam ocenených používateľov
        alebo príbuzný obsah z komunitných blogov\footnote{\url{https://meta.stackexchange.com/q/247298}}.}
  \item{\textbf{Aktuálnosť obsahu} -- Informačný bulletin niekedy obsahuje veľmi staré otázky, ktoré už nie sú
  relevantné\footnote{\url{https://meta.stackoverflow.com/q/319095}}.}
\end{my_itemize}

Na všetky tieto problémy používatelia upozorňujú už dlhšiu dobu, tieto otázky majú pomerne veľkú podporu komunity,
no napriek tomu žiaden z týchto problémov zatiaľ nebol adresovaný, a informačný bulletin ostáva aj naďalej generický
a z veľkej časti plný náhodného obsahu.

Generický informačný bulletin stráca pre používateľov informačnú hodnotu, pretože
najmä v prípade väčších komunít, akou je napríklad Stack Overflow\footnote{\url{https://stackoverflow.com}}, často obsahuje
otázky, ktoré nie sú z oblastí záujmu používateľa.


\begin{figure}[H]\begin{center}
\includegraphics[scale=0.45]{so-newsletter}
\caption{Informačný bulletin komunity Stack Overflow, 25. apríl 2017. Adresát tohto bulletinu má aktivitu prevažne v
PHP a SQL, čo vôbec nezodpovedá obsahu vygenerovaného bulletinu. \label{fig:so-newsletter}}\end{center}
\end{figure}

\subsection{Informačný bulletin portálu Quora}

Quora\footnote{\url{https://quora.com}} je CQA systém, ktorý nie je zameraný na konkrétnu oblasť záujmu, ale obsahuje
otázky z rôznych tém. Quora ponúka svojim používateľom týždenný informačný bulletin (\emph{Quora Weekly Digest}),
ktorý obsahuje desať najzaujímavejšich otázok za posledný týždeň a zoznam ľudí, ktorých používateľ potenciálne pozná.

Zoznam najzaujímavejších otázok pozostáva z editormi manuálne vybraného obsahu a algoritmicky vybraného obsahu,
ktorý je personalizovaný pre každého používateľa zvlášť~\footnote{\url{http://businessinsider.com/quora-emails-2012-8}\label{fnote-bi}} (Obr.~\ref{fig:quora-newsletter}).
Nie je však známe, akým spôsobom je vyberaná personalizovaná časť informačného bulletinu.

\begin{figure}[H]\begin{center}
\includegraphics[scale=0.55]{quora-newsletter}
\caption{Informačný bulletin portálu Quora. Prevzaté 30.4.2017,~[\ref{fnote-bi}] \label{fig:quora-newsletter}}\end{center}
\end{figure}

\subsection{CQA systémy bez informačných bulletinov}

Viaceré populárne CQA systémy svojim používateľom vôbec neponúkajú možnosť odoberať informačný bulletin. Medzi takéto
systémy patrí napr. portál \emph{Yahoo! Answers}\footnote{\url{https://answers.yahoo.com}}, ktorý je určený na pokladanie
otázok z akejkoľvek oblasti záujmu. Rovnako informačný newsletter neponúka ani ďalší všeobecne zameraný CQA systém --
\emph{Wiki Answers}\footnote{\url{https://answers.com}}. CQA systém zameraný na podporu výučby
\emph{Askalot}\footnote{\url{https://askalot.fiit.stuba.sk}} tiež v súčasnosti neponúka informačný newsletter,
iba možnosť notifikácie používateľa prostredníctvom e-mailu o aktivite súvisiacej s jeho obsahom v rámci systému.


\section{Diskusia}

CQA systémy patria medzi systémy s veľkým objemom používateľmi generovaného obsahu. Ako také sú vhodným kandidátom na
implementáciu informačných bulletinov. Existujúce informačné bulletiny v skúmaných CQA systémoch, najmä v prípade
platformy Stack Exchange, však majú veľké množstvo nedostatkov.

Riešením týchto problémov je vytváranie personalizovaných informačných bulletinov pre tieto systémy. Vhodným spôsobom
na personalizáciu je využitie metód odporúčania, ktorým sa venuje kapitola~\ref{rec}.



%%%%%%%%%%%%%%%%%%%%%%%%%%%%%%%%%%%%%%%%%%%%%%%%%%%%%%%%%%%%%%%%%%%%%%%%%%%%%%%%%%%%%%%%%%%%%%%%%%%%%%%%%%%%%%%%%%%%%%%%


\chapter{Odporúčanie}\label{rec}

\section{Odporúčacie systémy}

Odporúčacie systémy sú softvérové nástroje a techniky ktoré používateľom ponúkajú položky, ktoré by pre nich mohli
byť nejakým spôsobom zaujímavé alebo užitočné~\cite{Handbook2011}. Tieto odporúčania sú zvyčajne ponúkané za účelom
pomôcť používateľovi rozhodnúť sa, aké články by si mal prečítať, alebo aký tovar si kúpiť.

Využívanie odporúčacích systémov je tiež pre používateľov vhodným spôsobom, ako zvládať problémy informačného zahltenia
v dnešnom online svete. Ako také sa odporúčacie systémy stávajú jedným z najsilnejších a najpopulárnejších nástrojov
v online komunitách.

Odporúčacie systémy typicky vytvárajú zoznam odporúčaní jedným z dvoch spôsobov -- buď prostredníctvom \emph{kolaboratívneho
filtrovania} (angl.~\emph{Collaborative filtering}), alebo použitím \emph{filtrovania založeného na obsahu}
(angl.~\emph{Content-based filtering})~\cite{Buhmann2011}. Tieto dva prístupy môžu byť tiež kombinované
v hybridných odporúčacích systémoch.

\subsection{Kolaboratívne filtrovanie}\label{rec:collab}

Odporúčacie systémy využívajúce kolaboratívne filtrovanie fungujú prostredníctvom získavania spätnej väzby používateľa
vo forme hodnotení pre položky v~danej doméne a využívajú podobnosti v~hodnotení medzi viacerými používateľmi na určenie
či určitý obsah odporučiť, alebo nie~\cite{Buhmann2011}. Metódy kolaboratívneho filtrovania možno ďalej rozdeliť
na metódy založené na susednosti alebo na základe modelu.

Kolaboratívne filtrovanie na základe susednosti (angl.~\emph{Neighborhood-based Collaborative filtering})
vyberá skupinu používateľov podľa ich podobnosti k aktuálnemu používateľovi
a použitím váženej kombinácie ich hodnotení vyberá odporúčaný obsah pre tohto používateľa.
Techniky založené na modeli (angl.~\emph{Model-based Collaborative filtering}) poskytujú odporúčania prostedníctvom
oceňovania parametrov štatistických modelov pre používateľské hodnotenia.

\subsection{Filtrovanie na základe obsahu}\label{rec:content}

Odporúčanie čisto prostedníctvom kolaboratívneho filtrovania využíva iba používateľské hodnotenia. Tieto prístupy berú
všetkých používateľov a položky ako atomické jednotky a odporúčania sú vytvárané bez ohľadu na konkrétne špecifiká
individuálnych používateľov alebo položiek.

Metódy využívajúce filtrovanie na základe obsahu naopak vytvárajú
odporúčania na základe porovnávania modelov reprezentujúcich obsah s modelmi reprezentujúcimi konkrétneho používateľa~\cite{Handbook2011}.
Odporúčania v takýchto prístupoch vznikajú na základe prekryvu týchto dvoch modelov.


\section{Odporúčanie v CQA systémoch}

V kontexte CQA systémov je problematika odporúčania a odporúčacích systémov častým objektom výskumu~\cite{Srba2016}.

Jedným z hlavných cieľov CQA systémov je poskytnúť pýtajúcemu sa odpoveď na jeho otázku v čo možno najkratšom čase.
Rovnako ako v prípade iných systémov založených na používateľmi vytváranom obsahu, aj v prípade CQA systémov miera
nového obsahu -- nových otázok a odpovedí -- neustále narastá. Napriek tomu je tiež možné pozorovať stúpajúci trend
nízkej miery zodpovedanosti otázok~\cite{Srba2016SOFail}. Jedným zo spôsobov, ako je možné riešiť túto situáciu,
je práve využitie odporúčacích systémov.

V súčasnosti jedným z trendov najmä v úzko zameraných CQA systémoch ako napr. Stack Overflow, je tiež postupný prechod
od modelu jednoduchého odpovedania na položené otázky na model povzbudzujúci k vytváraniu dlhodobo hodnotného obsahu
vo forme rozsiahlych komunitne spravovaných odpovedí~\cite{Anderson2012,Li2015} podnecujúcich diskusiu.
V tomto prípade je možné využiť odporúčacie systémy ako prostriedok pre odhalenie a prezentovanie otázok a odpovedí,
ktoré by mohli používateľa zaujímať a priniesť mu úžitok aj v prípade, že práve nemá rovnaký problém,
ako sa vyskytuje v danej otázke~\cite{Toba2014}.

Výskum v oblasti odporúčania v CQA systémoch sa v súčasnosti zameriava hlavne na oblasti odporúčania, smerovania
a získavania otázok. Problémom súčasného výskumu v tejto oblasti je nejednoznačnosť a časté zamieňanie týchto výrazov,
prípadne nerozlišovanie medzi odporúčaním a smerovaním otázok~\cite{Srba2016}.

\subsection{Odporúčanie otázok}\label{q:rec}

Odporúčanie otázok (angl.~\emph{Question recommendation}) využíva tzv. \emph{pull} prístup, teda na základe (explicitnej
či implicitnej) požiadavky používateľa prezentuje zoznam odporúčaných relevantných otázok (alebo obsahu celkovo).
Tento prístup využíva štandardnejší tok medzi použitými modelmi -- začína sa modelom používateľa, na ktorý sa odporúčací
systém pokúša namapovať model relevantného obsahu.
Relevancia otázok pre používateľa môže byť identifikovaná rôznymi prístupmi -- či už na základe kolaboratívneho
filtrovania (kap.~\ref{rec:collab}) alebo filtrovania na základe obsahu (kap.~\ref{rec:content}).

Forma prezentovania odporúčaných otázok sa tiež môže líšiť. Časté je napríklad zobrazenie otázok, ktoré by používateľa
mmohli zaujímať, v detaile konkrétnej otázky, ktorú momentálne používateľ číta. Odporúčanie otázok je však možné využiť
aj ako prostriedok pre zvýšenie záujmu a angažovanosti používateľa o CQA systém.


\subsection{Smerovanie otázok}\label{q:routing}

Na rozdiel od pomerne štandardného odporúčania otázok, v prípade smerovania otázok (angl.~\emph{Question routing})
je prístup k odporúčaniu presne opačný, a využíva tzv. \emph{push} prístup. V tomto prípade proces odporúčania začína
modelom nezodpovedanej otázky, ktorú sa snaží odporúčací systém nasmerovať k používateľom, ktorí majú najväčší potenciál
na túto otázku zodpovedať.

Výskum smerovania nezodpovedaných otázok na konkrétnych používateľov -- odpovedajúcich -- síce ukazuje, že ide o dôležitý
koncept aj z pohľadu používateľského zážitku~\cite{Li2010,Li2011}, no prináša so sebou aj problémy. Najvýraznejším z nich je zahltenie
expertov, ktorí sú hlavnými terčmi takejto formy odporúčania, nakoľko ich reputácia a expertíza ich predurčuje ako vhodných
kandidátov na zodpovedanie veľkého množstva otázok~\cite{Pal2015}.

Pomerne novým prístupom k smerovaniu otázok je namiesto zamerania na konkrétnych používateľov smerovanie otázok na väčšie
komunity používateľov~\cite{Liu2014}. Hlavnou ideou takéhoto smerovania je fakt, že kolektívne poznatky komunity sú vždy väčšie, ako
poznatky konkrétneho používateľa, aj experta~\cite{Pal2013}. Navyše takéto smerovanie zvyšuje pravdepodobnosť rýchlejšieho zodpovedania
otázky, ako aj zabraňuje zahlteniu expertov. Hlavným problémom smerovania na komunity je vytváranie kolektívneho modelu
reprezentujúceho komunitu, kedy je potrebné brať do úvahy okrem iného fakt, že iba malá časť komunity sú \emph{tvorcovia
poznatkov} a nie len ich konzumenti~\cite{Pal2015}.


\subsection{Získavanie otázok}\label{q:retr}

Tento pojem (angl.~\emph{Question retrieval}) v kontexte CQA systémov hovorí o procese vyberania podobných otázok pre
rôzne formy dopytov~\cite{Zhang2014} na základe syntaktickej podobnosti otázok.
Tento proces je možné využiť na hľadanie odpovedí alebo poznatkov vo veľkých množstvách už zodpovedaných otázok.

Výskum v oblasti získavania otázok v kontexte CQA ststémov sa zameriava hlavne na problém premostenia lexikálnej bariéry medzi
obsahovo podobnými otázkami, ktoré sú však formulované použitím iných slov, synonymných výrazov a pod. Na prekonanie
tohto problému sa štandardne využíva vytvorenie prekladových modelov medzi otázkami a odpoveďami~\cite{Cao2010}.
Základnou myšlienkou za týmto postupom je predpoklad, že otázky a odpovede sú v podstate \emph{paralelnými textami}
a vzťahy medzi nimi môžu byť určené na základe pravdepodobností medzislovných prekladov.

Odlišný prístup k získavaniu otázok v CQA systémoch volia autori v~\cite{Zhang2014}. Argumentujú, že základný predpoklad
paralelnosti medzi otázkami a odpoveďami je v praxi nesprávny, pričom problémové sú hlavne odpovede veľmi nízkej kvality.
Autori preto navrhujú metódu tématicky-založeného jazykového modelu (angl.~\emph{Topic-based Language Model}), ktorá
predpokladá, že napriek tomu, že otázky a odpovede sú rozdielne vo viacerých aspektoch, zdieľajú určité
spoločné latentné faktory, ktoré predstavujú latentnú tému danej otázky a odpovede. Samotné získavanie otázok je následne
postavené práve na modeli, ktorý okrem lexikálnej podobnosti a prekladu berie do úvahy tieto latentné témy.


\subsection{Problém studeného štartu}\label{cold-start}

Častým problémom odporúčania je problém studeného štartu (angl.~\emph{Cold start}), kedy je na dosiahnutie primeranej
miery presnosti odporúčania potrebné veľké množstvo informácií, ktoré ale napr. v prípade nových alebo menej aktívnych
používateľov nemusia byť k dispozícii.

Tento problém sa vyskytuje najmä v prípade systémov, ktoré obsah odporúčajú na základe
podobnosti používateľov medzi sebou. Keďže je často na začiatok potrebné veľké množstvo informácií o daných používateľoch,
nie je možné jednoducho odporúčať vhodný obsah pre používateľov, ktorí sú menej aktívni, alebo sú noví.
V menšej miere týmto problémom trpia systémy, ktoré namiesto podobnosti používateľov využívajú pre zostavovanie odporúčaní
podobnosť samotného obsahu na základe rôznych atribútov.

V kontexte odporúčania otázok v CQA systémoch je tento problém úzko previazaný s problémom dlhého chvosta v používateľskej aktivite (kapitola~\ref{cqa:tail}).
Veľmi veľké percento používateľskej základne tvoria používatelia, ktorí sú noví, alebo nemajú žiadnu aktivitu.
Pre odporúčanie otázok (kapitola~\ref{q:rec}) je tak problém zostaviť profil používateľa, na základe ktorého sa vykonáva
mapovanie na model relevantného obsahu. V prípade smerovania otázok (kapitola~\ref{q:routing}) je zase problém zostaviť
model reprezentujúci expertízu daného používateľa.

\subsection{Problém filtračnej bubliny}\label{rec:filterbubble}

Ďalším problémom, ktorému sa však v oblasti odporúčania obsahu CQA systémov venuje menej pozornosti~\cite{Srba2016},
je rôznorodosť odporúčaného obsahu. Hrozí tak výskyt problému tzv. filtračnej bubliny (angl.~\emph{Filter bubble}).

Ak totiž systém používateľovi odporúča obsah len z oblastí používateľovho záujmu, dochádza k problému, kedy je používateľ
do značnej miery uzatvorený v rámci jednej oblasti a nemá tak možnosť získavať zaujímavé poznatky z iných oblastí.
Používateľ sa tak síce môže stať odborníkom na danú oblasť, no jeho povedomie o~širšom kontexte celej problematiky
je veľmi obmedzené.

Riešením tohto problému je identifikácia oblastí, ktoré nie sú priamo oblasťami záujmu používateľa, no sú k týmto
oblastiam v určitých aspektoch príbuzné. Používateľ má tak možnosť rozšíriť svoj okruh záujmu a vedomosti o širšom
kontexte problémovej domény.


% \section{Existujúce metódy pre odporúčanie otázok v CQA}

% \textbf{TODO}

% \subsection{Črty a algoritmy}

% Survey

% \subsection{Metodológie overenia}

% Survey

% \subsection{Problém studeného štartu}
% - Srba - Utilizing non-QA data to improve questions routing for users with low QA activity in CQA
% - Cold-Start Expert Finding in Community Question Answering via Graph Regularization

% \subsection{Problém filtračnej bubliny}
% - Survey
% - Szpektor - diverzifikacia

% \section{Diskusia}
% TODO


%%%%%%%%%%%%%%%%%%%%%%%%%%%%%%%%%%%%%%%%%%%%%%%%%%%%%%%%%%%%%%%%%%%%%%%%%%%%%%%%%%%%%%%%%%%%%%%%%%%%%%%%%%%%%%%%%%%%%%%%


\chapter{Diverzita a aktuálnosť odporúčania}

Diverzitu možno všeobecne definovať ako opak podobnosti. V niektorých prípadoch však nemusí byť odporúčanie podobných
položiek tým najlepším riešením pre používateľa~\cite{Handbook2011}. Dôvodom je práve náchylnosť takéhoto odporúčania na
vyvolanie problému filtračnej bubliny (viď. kapitola \ref{rec:filterbubble}).

Okrem diverzifikácie odporúčaného obsahu má na celkovú úspešnosť vytvárania personalizovaných odporúčaní veľký vplyv aj
aktuálnosť (angl.~\emph{freshness}, príp.~\emph{novelty}) odporúčaného obsahu~\cite{Liu2015}.

\section{Diverzita v odporúčacích systémoch}

Tématická diverzifikácia je metóda napomáhajúca vyváženosti a diverzite personalizovaného odporúčania s cieľom
lepšie reflektovať kompletné spektrum používateľových záujmov. Napriek tomu, že môže mať negatívny vplyv na priemernú
správnosť odporúčaní, dosahuje táto metóda zvýšenú úroveň používateľskej spokojnosti~\cite{Zhang2009}.

\textbf{Diverzita na základe nízkej vnútornej podobnosti zoznamu}\\
Ziegler a kol.~\cite{Ziegler2005} študovali diverzifikáciu v oblasti odporúčaní a navrhli prístup, ktorý vytvára
zoznamy odporúčaní lepšie uspokojujúce používateľove záujmy prostredníctvom selekcie takých zoznamov, ktoré majú nízku
vnútornú podobnosť.

Odporúčanie bolo navrhnuté s použitím kolaboratívneho filtrovania na základe položiek (kapitola~\ref{rec:collab}).
Vnútorná miera podobnosti položiek zoznamov bola určovaná prostredníctvom metriky založenej na taxonomickej klasifikácií
jednotlivých položiek. Samotná diverzifikácia spočívala v sekvenčnom výbere položiek z kandidátnych zoznamov odporúčaní tak,
aby bola minimalizovaná vnútorná tématická podobnosť výsledného zoznamu odporúčaní.
Autori v online experimente demonštrovali, že reálni používatelia preferujú diverznejšie výsledky.


\textbf{Diverzita na základe dôvodu}\\
Tradičný spôsob zavedenia diverzity do odporúčania je diverzifikácia na základe atribútov odporúčaného obsahu, teda
zoskupenie výsledkov do skupín zdieľajúcich viaceré atribúty (ako napr. žáner hudby) a následný výber iba limitovaného
množstva výsledkov z každej zo skupín. Autori v~\cite{Yu2009} prezentujú \emph{diverzifikáciu na základe dôvodu}.
Táto metóda využíva pre diverzifikáciu výsledkov dôvod, prečo bola konkrétna položka odporučená
(napr. \textit{tento album bol odporučený, pretože ste počúvali inú skladbu tohto autora}).
V článku autori nekonkretizujú spôsob určenia dôvodov pre odporúčanie, len formálne definujú metriku pre výpočet diverznosti
medzi odporúčanými položkami ako priemer kosínovej vzdialenosti medzi vektormi reprezentujúcimi tieto dôvody.
Pre aplikovanie diverzifikácie v odporúčaní \emph{top-K} položiek využívajú nasledovný algoritmus:

\begin{adjustwidth}{1cm}{1cm}
Pre daného používateľa $u$ a prah $\theta$ pre agregované skóre množiny odporúčaných položiek $\sigma$, nájdi množinu
$S \subseteq \sigma$ takú, že $|S| = k, score(S) >= \theta$ a priemerná kosínová vzdialenosť diverzity položiek je maximalizovaná.
\end{adjustwidth}

Diverzifikáciu na základe dôvodu autori porovnávajú so štandardným prístupom diverzifikácie na základe atribútov odporúčaného obsahu.
Autori experimentálne ukázali, že takáto forma diverzifikácie je prinajmenšom rovnako účinná, ako diverzifikácia na
základe atribútov položiek, pričom z pohľadu výkonu ju výrazne presahuje.

\textbf{Diverzita na báze proporcionality}\\
Inú perspektívu volí metóda diverzity na báze proporcionality. Zoznam odporúčaní je možné považovať za najlepšie diverzifikovaný
vzhľadom na relevanciu odporúčaní v takom prípade, keď počet výsledkov z určitej témy je úmerný popularite danej témy.
Dang a kol. vo svojej práci~\cite{Dang2012} ponúkajú koncept optimalizácie proporčnosti pre diverzifikáciu výsledkov
vyhľadávania.

Napriek tomu, že sa práca~\cite{Dang2012} nezaoberá priamo diverzitou v odporúčacích systémov, sú paralely s touto
oblasťou výrazné. Motiváciou pre tákyto spôsob diverzifikácie je metóda obsadzovania kresiel v parlamente. Ich metóda
postupne pre každú pozíciu v zozname výsledkov určuje tému, ktorá najlepšie zachováva celkovú proporčnosť. Následne
na túto pozíciu z danej témy vyberie najlepší dokument.


\section{Aktuálnosť v odporúčacích systémoch}

Aktuálnosť v kontexte odporúčacích systémov môže predstavovať dva rôzne aspekty. Jedným z nich je \emph{novosť}
(angl.~\emph{novelty}), teda pomerne priamočiara vlastnosť určujúca, či odporúčaný obsah už bol používateľovi prezentovaný,
alebo nie. Jednoduchým spôsobom, ako zabezpečiť, aby používateľovi neboli stále dookola odporúčané tie isté položky,
akokoľvek relevantné by pre neho mohli byť, je odfiltrovanie položiek, ktoré už používateľovi boli odporúčané v minulosti,
a s ktorými už interagoval~\cite{Handbook2011}.

Druhým aspektom aktuálnosti je \emph{čerstvosť} (angl.~\emph{freshness}). V tomto prípade ide o časovú aktuálnosť odporúčaného
obsahu. Naivný prístup k aktuálnosti je odporúčanie iba obsahu z určitého obmedzeného časového úseku z blízkej minulosti,
no takýto prístup nemusí vždy dosahovať najlepšie výsledky používateľskej spokojnosti~\cite{Szpektor2013}.

Pri odporúčaní, ktoré berie do úvahy aktuálnosť odporúčaného obsahu je dôležitým aspektom detekcia časovo citlivej témy.
Takýto systém by mal presadzovať aktuálny obsah iba v prípade, kedy je to vhodné. Na druhej strane, ak je v prípade
aktívne sa vyvýjajúcej témy odporúčaný neaktuálny obsah, môže to výrazne degradovať úspešnosť odporúčania~\cite{Dong2010}.
Ďalším faktorom pri posudzovaní aktuálnosti v odporúčaní je časová škála aktuálnosti pre danú tému.
V prípade niektorých tém alebo oblastí je možné považovať za aktuálne položky z posledného roka, no v iných prípadoch
môžu byť aj niekoľko týždňov staré položky považované za vysoko neaktuálne.

Ďalším problémom vzhľadom na aktuálnosť v odporúčaní je tiež fakt, že pre novo vzniknutý obsah môže byť problémovejšie
zostaviť model, ktorý by ho reprezentoval, nakoľko môže byť o tomto obsahu známych zatiaľ iba málo informácií~\cite{Dong2010TW}.
Tento problém studeného štartu sa môže prejavovať rovnako v prípade modelovania obsahu, ako je tomu napríklad v prípade
vytvárania modelov reprezentujúcich nových používateľov.

Riešením v takomto prípade môže byť napríklad vytváranie modelu
obsahu na základe čŕt, ktoré nie sú ovplyvnené časom (napr. nadpis, text alebo autor obsahu, na rozdiel od počtu hlasov
alebo dátumu uverejnenia), alebo tiež upravenie hodnôt týchto čŕt vzhľadom na relatívnu aktuálnosť obsahu.


\section{Diverzita a aktuálnosť v kontexte CQA systémov}

Kým diverzita a aktuálnosť sú v oblasti odporúčania a celkovo vo vyhľadávaní informácií pomerne často analyzovanými aspektmi,
v kontexte CQA systémov sa týmto hľadiskám doteraz venovala iba okrajová pozornosť~\cite{Srba2016}.

Liu a kol. vo svojej práci~\cite{Liu2015} skúmajú aspekt aktuálnosti odporúčania v CQA systémoch prostredníctvom
relatívne neštandardného návrhu CQA systému určeného pre odpovedanie v reálnom čase na \emph{hyper-lokálne} a časovo senzitívne
otázky. Za týmto účelom využívajú prístupy predchádzajúcich prác a kombinujú aspekty relevancie, lokality a aktuálnosti
v \emph{real-time} CQA systéme.


Komplexnejší pohľad na aktuálnosť a diverzitu priamo v kontexte štandardných CQA systémov ponúka Szpektor a kol~\cite{Szpektor2013}.
Autori experimentovali so zavedením diverzity a aktuálnosti do procesu vytvárania odporúčaní pre používateľov CQA
systému Yahoo! Answers, pričom sa na rozdiel od väčšiny prác v tejto oblasti nezameriavali len na skupinu expertných používateľov.

Pre odporúčanie využili profil otázok založený na kombinácii LDA, lexikálneho a kategorického modelu a profil používateľa
odvodený od profilu otázok, s ktorými interagoval. Párovanie otázok a používateľov bolo vykonané prostredníctvom jednoduchého
skalárneho súčinu vektorov reprezentujúcich profily používateľov a~otázok.

Samotná diverzifikácia odporúčaní bola vykonávaná prostredníctvom tématického výberu vzoriek (angl.~\emph{thematic sampling}),
kedy je vygenerovaných viacero samostatných zoznamov odporúčaných otázok z viacerých tém, ktoré sú následne zmiešané dokopy
proporcionálne k pravdepodobnostnému skóre jednotlivých tématických zoznamov.

Prínos aktuálnosti do odporúčania v CQA systémoch bol skúmaný na základe odporúčania iba aktuálneho obsahu -- konkrétne
iba nezodpovedaných otázok za posledné štyri hodiny.

Dopad diverzifkácie a aktuálnosti na úspešnosť odporúčania bol testovaný v rámci online experimentu. Používatelia boli náhodne
rozdelení do štyroch segmentov:

\begin{my_enumerate}
  \item{\textbf{Kontrolná vzorka} -- Týmto používateľom neboli ponúknuté žiadne odporúčania.}
  \item{\textbf{Odporúčanie na základe relevancie} -- Používateľom boli ponúknuté odporúčania iba na základe relevancie
        daných otázok, bez ohľadu na aktuálnosť alebo diverzitu.}
  \item{\textbf{Odporúčanie s ohľadom na aktuálnosť} -- Používateľom boli odporúčané relevantné otázky, pričom 50\% z nich
        pochádzalo z posledných štyroch hodín a 20\% bolo vybraných prostredníctvom tématického výberu vzoriek.}
  \item{\textbf{Diverzifikované odporúčanie} -- Používateľom boli odporúčané relevantné otázky, pričom 50\% z nich
        bolo vybraných na základe tématického výberu vzoriek ako prostriedku diverzifikácie, a 20\% pochádzalo z posledných
        štyroch hodín.}
\end{my_enumerate}

Výsledky online experimentu potvrdili intuitívnu myšlienku, že iba samotná relevancia nie je dostatočná na úspešne
odporúčanie otázok v CQA systéme. Práve naopak, vo vykonanom experimente dokonca samotné odporúčanie len na základe
relevancie dosiahlo nižšie hodnoty zodpovedania otázok, ako kontrolná vzorka bez akýchkoľvek odporúčaní.

Presadzovanie aktuálnych otázok dosiahlo zvýšenie miery zodpovedania otázok o 4\%, avšak najlepšie výsledky boli dosiahnuté
prostredníctvom diverzifikácie odporúčaní aj za cenu zníženia aktuálnosti, pričom miera zodpovedania sa zvýšila o 17\%.

\section{Diskusia}

Na základe tejto analýzy môžeme usúdiť, že napriek tomu, že problematika diverzifikácie a aktuálnosti odporúčania v kontexte
CQA systémov v súčasnosti stále ostáva do veľkej miery nepreskúmaná, je očividné, že uvažovanie týchto aspektov v tomto
kontexte má veľký vplyv na úspešnosť odporúčania, pričom informačné bulletiny sa javia ako prirodzená, požadovaná,
no napriek tomu málo využívaná forma prinášania odporúčaného a potenciálne zaujímavého obsahu používateľom.
