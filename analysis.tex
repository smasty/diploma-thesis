%!TEX root = main.tex

\newpage
\chapter{Informačné bulletiny}

Informačné bulletiny (angl. \emph{newsletters}) sú aj v súčasnosti jedným z najrozšírenejších spôsobov, ako v online
prostredí informovať používateľov o dianí na webovom portáli. Prevádzkovatelia webových portálov využívajú informačné
bulletiny na predstavenie nového obsahu, akciového tovaru, zaujímavostí z určitej oblasti alebo špeciálnych ponúk pre
svojich používateľov a zákazníkov.

Informačné bulletiny spravidla nadobúdajú formu e-mailu, ktorý je zvyčajne v pravidelných intervaloch doručovaný
do schránok používateľov, ktorí o jeho doručovanie prejavili záujem.


\section{Problémy informačných bulletinov}
Hlavným problémom informačných bulletinov je interakcia používateľov s informačnými bulletinmi.

Štúdia spoločnosti Silverpop z roku 2012\cite{mailmarketing} na vzorke informačných bulletinov 1124 spoločností ukázala,
že počet používateľov, ktorí vôbec otvoria informačný bulletin sa pohybuje na úrovni 20\% a stále klesá. Navyše konkrétne
v oblasti technológií sa táto hodnota pohybuje len na 16,5\%. Ešte menšia je miera preklikov
(angl. \emph{Click-through rate - CTR}), ktorá sa celkovo pohybuje na úrovni 5,4\% a v prípade technologicky zameraných
informačných bulletinov len 3,6\%. Napriek tomu sa miera odhlásení z odoberania (angl. \emph{unsubscribe rate}) pohybuje
len na úrovni 2\%.

Dôvodov, prečo používatelia prejavujú iba malý záujem o informačné bulletiny ktoré im sú doručované, môže byť niekoľko.
Jedným z takýchto dôvodov môže byť vysoká saturácia - používateľom chodí priveľké množstvo informačných bulletinov,
dôsledkom čoho používatelia rezignujú a tieto e-maily ani neotvárajú.
Hlavným nedostatkom informačných bulletinov, a zároveň dôvodom, prečo iba 5\% používateľov klikne na obsah v informačnom
bulletine, je však relevancia ponúkaného obsahu.


\section{Relevancia v informačných bulletinoch}

Množstvo webových portálov doručuje všetkým svojim používateľom presne ten istý obsah informačného bulletinu.
Často je tento obsah vytváraný manuálne editormi, a zameriava sa len všeobecne na aktuálne dianie na danom webovom
portáli. Takýto všeobecný informačný bulletin však nutne nemôže byť dostatočne relevantný pre značnú časť používateľov.

Riešením problému relevancie infromačných bulletinov je vytváranie personalizovaných informačných bulletinov, ktoré
každému používateľovi ponúkajú len ten obsah, ktorý je pre neho najzaujímavejší a najrelevantnejší.


\section{Informačné bulletiny v CQA systémoch}

ako newslettery vyuzivaju CQA systemy - nevnejue sa tomu pozornost (citovat Srbu)

porovnat newsletterty Stack Ovetrflow, Quora a Y!A



%%%%%%%%%%%%%%%%%%%%%%%%%%%%%%%%%%%%%%%%%%%%%%%%%%%%%%%%%%%%%%%%%%%%%%%%%%%%%%%%%%%%%%%%%%%%%%%%%%%%%%%%%%%%%%%%%%%%%%%%


\newpage
\chapter{CQA systémy}

    - Co to je
    - Cim sa lisia od inych user-generated content sites
    - Rozdelenie CQA systemov - Yahoo Answers-type vs SO-type



%%%%%%%%%%%%%%%%%%%%%%%%%%%%%%%%%%%%%%%%%%%%%%%%%%%%%%%%%%%%%%%%%%%%%%%%%%%%%%%%%%%%%%%%%%%%%%%%%%%%%%%%%%%%%%%%%%%%%%%%


\chapter{Odporúčanie}

- Odporucacie systemy celkovo - o co ide (len kratko)
    - Content-based filtering
    - Collaborative filtering
- Odporucanie v CQA systemoch
- Question Routing, Question recommendation, retrieval... definitions...
- Challenges:
    - Cold start
    - Filter bubble



%%%%%%%%%%%%%%%%%%%%%%%%%%%%%%%%%%%%%%%%%%%%%%%%%%%%%%%%%%%%%%%%%%%%%%%%%%%%%%%%%%%%%%%%%%%%%%%%%%%%%%%%%%%%%%%%%%%%%%%%


\chapter{Diverzifikácia a aktuálnosť}

- Diverzifikacia v odporucacich systemoch -> riesenie filter bubble
- Dopad aktualnosti na uspesnost odporucania
- D and A v CQA
