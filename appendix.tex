%!TEX root = main.tex
\newpage
\thispagestyle{plain}

\chapter{Plán práce na diplomovom projekte}

\section{Plán práce na diplomovom projekte I}

Prácu na diplomovom projekte I sme navrhli a naplánovali nasledovne:

\begin{table}[h]
\centering
\caption{Plán práce na diplomovom projekte I}
\begin{tabular}{|m{2.3cm}|m{12cm}|}
\hline
1-5.~týždeň semestra   & Analýza problematiky odporúčania v kontexte CQA systémov a súčasného výskumu v tejto oblasti. \\ \hline
6-7.~týždeň semestra   & Analýza dostupných dát na platforme Stack Exchange a možností verejného API platformy. \\ \hline
8-9.~týždeň semestra   & Vytvorenie predbežného návrh metódy riešenia problému a návrh metód overenia výsledkov. \\ \hline
10-12.~týždeň semestra & Spísanie správy diplomového projektu I v rátane analýzy a predbežného návrhu riešenia a overenia. \\ \hline
\end{tabular}
\end{table}

\textbf{Zhodnotenie}\\
Navrhnutý plán práce v rámci diplomového projektu I sa nám do veľkej miery podarilo dodržať. Časový sklz sa prejavil až
vo fáze návrhu metódy riešenia problému, čím sa posunulo spísanie správy až do 11. týždňa semestra.

\newpage
\section{Plán práce na diplomovom projekte II}

\begin{table}[h]
\centering
\caption{Plán práce na diplomovom projekte II}
\begin{tabular}{|m{2.3cm}|m{12cm}|}
\hline
1-2.~týždeň semestra   & -- Príprava databázy a aktualizačného modulu.
				\newline -- Vytvorenie modelov pre získavanie spätnej väzby od používateľov.
				\newline -- Prvotná dátová analýza.
				\\ \hline
3-4.~týždeň semestra   & -- Príprava rozhrania pre odoberanie informačných bulletinov.
				\newline -- Generovanie informačných bulletinov na základe prvotného modelu.
				\newline -- Nasadenie systému a otestovanie v reálnych podmienkach.
				\\ \hline
5-8.~týždeň semestra   & -- Získavanie používateľov informačného bulletinu.
				\newline -- Spracovanie dát, vytvorenie reálnych modelov používateľov a otázok, generovanie odporúčaní.
				\newline -- Príprava metód diverzifikácie odporúčaní.
				\newline -- Písanie diplomovej práce.
				\\ \hline
9-12.~týždeň semestra  & -- Overenie a vyhodnocovanie informačných bulletinov.
				\newline -- Nasadenie a porovnanie metód diverzifikácie.
				\newline -- Počiatočné overenie výsledkov.
				\newline -- V prípade neúspechu online experimentu, plánovanie offline expermientov.
				\\ \hline
\end{tabular}
\end{table}

\section{Plán práce na diplomovom projekte III}

\begin{table}[h]
\centering
\caption{Plán práce na diplomovom projekte III}
\begin{tabular}{|m{2.3cm}|m{12cm}|}
\hline
1. mesiac & Pokračovanie v experimentoch, nasadzovanie systému na väčšom množstve dát. \\ \hline
2. mesiac & Analýza výsledkov experimentov, vyhodnotenie úspešnosti projektu. \\ \hline
3. mesiac & Dokončenie diplomovej práce.\\ \hline
\end{tabular}
\end{table}


\section{Technická dokumentácia systému StackLetter}\label{tech-doc}
