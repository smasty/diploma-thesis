%!TEX root = main.tex
\newpage
\thispagestyle{plain}
\begin{center}
\begin{Large}
\textbf{Anotácia} \\
\end{Large}
\end{center}
Slovenská technická univerzita v Bratislave
\vspace*{2mm}\\FAKULTA INFORMATIKY A INFORMAČNÝCH TECHNOLÓGIÍ
\vspace*{2mm}\\
\noindent
Študijný program:~\Program
\vspace*{2mm}\\
\noindent
Autor:\hspace*{21mm}\Author
\vspace*{2mm}\\
\ifthenelse {\boolean{bachelor}}
{
	{Bakalárska práca: }\Title
}
{
	{Diplomová práca:\hspace*{2mm}}Podpora diverzity a~aktuálnosti informačných bulletinov v~systéme\\
    \hspace*{32mm}pre~odpovedanie na~otázky
}
\vspace*{2mm}\\
Vedúci práce:\hspace*{9mm}\Supervisor
\vspace*{2mm}\\\Month \Year \\
\noindent
\\
Informačné bulletiny predstavujú štandardný spôsob ako informovať používateľov v online komunitách o novom alebo zaujímavom
obsahu. Ich význam je ešte väčší v online komunitách ktoré produkujú veľké množstvo používateľmi vytvoreného obsahu,
akými sú aj systémy pre odpovedanie na otázky.
Napriek tomu mnohé populárne CQA systémy ponúkajú iba generické informačné bulletiny, ktoré nijakým spôsobom nereflektujú
záujmy používateľa alebo diverzitu odporúčaného obsahu.


Cieľom našej práce je analyzovať existujúce prístupy k personalizovanému odporúčaniu v CQA systémoch a navrhnúť metódu
automatického vytvárania personalizovaných informačných bulletinov pre jednotlivých používateľov. Zameriavame sa na zlepšenie
diverzity a aktuálnosti odporúčaného obsahu ako spôsobu prevencie vzniku \emph{filtračnej bubliny} a zvýšenie celkovej
spokojnosti používateľov a ich interakcie s informačným bulletinom.

\afterpage{\blankpage}
\newpage
\thispagestyle{plain}
\begin{center}
\begin{Large}
\textbf{Annotation} \\
\end{Large}
\end{center}
Slovak University of Technology Bratislava
\vspace*{2mm}\\FACULTY OF INFORMATICS AND INFORMATION TECHNOLOGIES
\vspace*{2mm}\\
\noindent
Degree Course:~\ProgramEN
\vspace*{2mm}\\
\noindent
Author:\hspace*{14.5mm}\AuthorEN
\vspace*{2mm}\\
\ifthenelse {\boolean{bachelor}}
{
	{Bachelor thesis: }\TitleEN
}
{
	{Master thesis:\hspace*{4mm}}Improving Diversity and~Freshness of~Newsletters in~Community Question\\
    \hspace*{27.5mm}Answering Systems
}
\vspace*{2mm}\\
Supervisor: \hspace*{7mm}\SupervisorEN
\vspace*{2mm}\\\Year, \MonthEN\\
\noindent
\\
Newsletters represent a standard way to inform users of online communities about new or~interesting content.
Their importance is even greater in online communities producing large amounts of user-created data, such as
Community Question Answering systems.
Nevertheless, many popular CQA systems only offer generic newsletters, which do not take into account users’ interests
or diversity of the recommended content.

The aim of this work is to analyze existing approaches in personalized content recommendation in CQA systems and design a method for automatic
creation of personalized newsletters for individual users of CQA systems. We want to focus on improving the diversity and freshness of~the
recommended content as a way to prevent \emph{filter bubbles} and improve overall user satisfaction and engagement with the newsletter.

\afterpage{\blankpage}
