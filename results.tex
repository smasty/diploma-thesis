%!TEX root = main.tex
\afterpage{\blankpage}
\newpage
\chapter{Experimentálne overenie}
\label{sec:experiment}

Navrhnutú metódu vytvárania personalizovaných informačných bulletinov sme sa rozhodli overiť prostredníctvom
nekontrolovaného online experimentu. Väčšina prác venujúcich sa odporúčaniu v doméne CQA systémov overuje
svoje hypotézy v offline experimentoch na kontrolovaných vzorkách dát. Pre účely realistického overenia našej metódy, hlavne
z pohľadu faktorov špecifických pre informačné bulletiny, však overovanie v offline experimente nie je dostačujúce.
Online nekontrolované experimenty samozrejme vyžadujú oveľa viac námahy a ich výsledky nie sú garantované, no v prípade,
že je takýto experiment úspešný, má oveľa väčší potenciál priniesť relevantné výsledky.

Experiment sme vykonávali na používateľoch z komunity Stack Overflow platformy Stack Exchange.
Komunitu Stack Overflow sme pre náš experiment zvolili z viacerých dôvodov, hlavne však preto,
že sa jedná o jednoznačne najaktívnejšiu komunitu z celej platformy, v ktorej bol najväčší potenciál získať čo najviac
používateľov do nášho experimentu. Okrem toho zohrával úlohu v rozhodovaní aj fakt, že táto komunita sa venuje doménovej
oblasti, v ktorej máme určité vedomosti, čo nám umožňilo jednoduchšie odhadnúť úroveň relevancie odporúčaného obsahu.

Do experimentu sa mohli prihlásiť všetci používatelia, ktorí majú konto na portáli Stack Overflow, prostredníctvom
registračného formuláru systému StackLetter. Celkovo sa nám podarilo získať pomerne diverznú vzorku používateľov,
medzi ktorými boli prítomní okrem iných aj úplni nováčikovia bez akejkoľvek predošlej aktivity, ale aj mimoriadne aktívni
a dlhodobo rešpektovaní členovia komunity. Používatelia mali pri registrácii možnosť vybrať si buď denný alebo týždenný
informačný bulletin. Túto voľbu tiež mohli kedykoľvek počas experimentu zmeniť.

\textbf{TODO} -- GRAF - podiel daily/weekly subscriberov.

\section{Metodológia experimentu}
Experiment prebiehal formou A/B testovania. Na rozdiel od štandardného A/B testovania, v ktorom sa používatelia rozdelia
do dvoch alebo viacerých skupín, sme náš test vykonávali na báze striedania použitých metód v časových intervaloch:

\begin{adjustwidth}{1cm}{1cm}
\textbf{1. Kontrolná vzorka}\\
V prvej fáze experimentu, od 9. novembra 2017 do 11. marca 2018, sme používateľom generovali informačný bulletin len
prostredníctvom triviálnej metódy odporúčania, ktorá vyberala otázky označené značkami, v ktorých v minulosti používateľ
položil otázku, ponúkol odpoveď alebo okomentoval príspevok. Dáta z tejto skupiny slúžili ako počiatočné dáta pre porovnanie
s ostatnými metódami.

Od 12. marca 2018 sme nasadili metódy vytvárania odporúčaní opisované v tejto práci. Metódy generovania informačných bulletinov
sa striedali na týždennej báze.

\textbf{2. Metóda A -- Personalizované odporúčanie}\\
Využívala sa metóda personalizovaného odporúčania opísaná v kapitole~\ref{impl:pers-method}, pričom sa aplikovala
metóda pre zabezpečenie aktuálnosti odporúčaného obsahu. V tomto prípade sa nevykonávala žiadna diverzifikácia odporúčaní.

\textbf{3. Metóda B -- Personalizované odporúčanie s diverzifikáciou}\\
V tomto prípade sa zostavovali personalizované informačné bulletiny s využitím diverzifikačnej metódy tématického
vzorkovania, ako aj metódy pre zabezpečenie aktuálnosti odporúčaného obsahu.
\end{adjustwidth}

Bla bla bla.


% \section{Návrh overenia metód}

% Nami navrhnuté metódy personalizovaného odporúčania a diverzifikácie odporúčaného obsahu budeme overovať prostredníctvom
% online nekontrolovaného experimentu spoločne so spolužiakom Matúšom Salátom~\cite{Salat2018} na používateľoch z komunity
% \textit{Stack Overflow} platformy Stack Exchange.

% \textbf{TODO} - vysvetlit preco zrovna SO.

% Tento online experiment bude mať formu pravidelne rozposielaného informačného bulletinu, na ktorého odoberanie sa budú
% môcť prihlásiť všetci používatelia z tejto komunity.

% Účinnosť zvolených metód diverzifikácie odporúčaní budeme vyhodnocovať prostredníctvom A/B testovania, pričom používateľov
% rozdelíme na štyri skupiny:
% \begin{my_enumerate}
%     \item{Kontrolná skupina -- generický informačný bulletin bez odporúčania;}
%     \item{Skupina A -- informačný bulletin s odporúčaním, bez diverzifikácie;}
%     \item{Skupina B -- informačný bulletin s odporúčaním, metóda proporčnej diverzifikácie;}
%     \item{Skupina C -- informačný bulletin s odporúčaním, metóda tématického vzorkovania;}
% \end{my_enumerate}

% Dôležitým predpokladom pre úspešnosť online experimentu bude získať dostatočnú reprezentatívnu vzorku používateľov
% ochotných byť súčasťou experimentu. V~prípade, že by sa nám nepodarilo osloviť dostatočný počet používateľov, plánujeme
% vykonať kontrolovaný offline experiment na vzorke archívnych dát.


% \section{Metriky hodnotenia výsledkov}

% Pre overovanie výsledkov experimentov budeme používať nasledovné metriky:

% \textbf{Precision@N}\\
% Presnosť (angl.~\emph{Precision}), alebo tiež \textit{pozitívna predikčná hodnota} je metrika reprezentujúca pomer relevantných
% dokumentov z celkového zoznamu. Štandardne sa presnosť počíta ako pomer z celého zoznamu dokumentov, no v oblasti
% odporúčania a vyhľadávania informácií je často vhodnejšou odvodená metrika \textit{Precision@N}, ktorá určuje, aká časť
% z prvých N dokumentov v~zozname odporúčaní je pre používateľa relevantná.
% $$\mbox{Precision@N}=\frac{|\{\mbox{relevantne otazky v top-N}\}\cap\{\mbox{top-N odporucenych otazok}\}|}
% {|\{\mbox{top-N odporucenych otazok}\}|}$$

% \textbf{DCG}\\
% \textit{Discounted Cumulative Gain} je metrika kvality ohodnocovania často používaná na meranie efektívnosti
% odporúčania. DCG meria užitočnosť dokumentov na základe ich pozície vo výslednom zozname. Užitočnosť dokumentov sa akumuluje
% od konca zoznamu, pričom najvyššiu užitočnosť majú dokumenty na začiatku zoznamu~\cite{Jrvelin2002}.

% DCG položky na pozícii $p$ v zozname odporúčaní je definované ako:

% $$\mathrm{DCG_{p}} = \sum_{i=1}^{p} \frac{ 2^{rel_{i}} - 1 }{ \log_{2}(i+1)}$$
% \begin{adjustwidth}{1cm}{1cm}
% $rel_i$ -- relevancia $i$-tej položky v zozname odporúčaní.\\
% \end{adjustwidth}

% \textbf{CTR}\\
% Miera preklikov (angl.~\emph{Click-through Rate}) je metrika často využívaná v spojitosti s informačnými bulletinmi.
% Táto metrika vyjadruje počet úspešných kliknutí na odkaz v informačnom bulletine.

% Okrem CTR plánujeme v súvislosti s interakciou používateľa s informačným bulletinom merať aj počet impresií,
% teda zobrazení informačného bulletinu, ako aj počet konverzií, teda podiel prípadov, kedy kliknutie na niektorú z otázok
% v informačnom bulletine viedlo k aktivite používateľa na tejto otázke -- či už označenie za obľúbenú,
% odpovedanie alebo pridanie komentáru. Ďalej tiež plánujeme merať počet odhlásení z informačného bulletinu.
