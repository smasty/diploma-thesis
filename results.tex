%!TEX root = main.tex
\newpage
\chapter{Experimentálne overenie}


\section{Návrh overenia metód}

Nami navrhnuté metódy personalizovaného odporúčania a diverzifikácie odporúčaného obsahu budeme overovať prostredníctvom
online nekontrolovaného experimentu spoločne s kolegom Matúšom Salátom~\cite{Salat2018} na používateľoch z komunity \textit{Stack Overflow} platformy Stack Exchange.

Tento online experiment bude mať formu pravidelne rozposielaného informačného bulletinu, na ktorého odoberanie sa budú
môcť prihlásiť všetci používatelia z tejto komunity.

Účinnosť zvolených metód diverzifikácie odporúčaní budeme vyhodnocovať prostredníctvom A/B testovania, pričom používateľov
rozdelíme na štyri skupiny:
\begin{my_enumerate}
    \item{Kontrolná skupina -- generický informačný bulletin bez odporúčania;}
    \item{Skupina A -- informačný bulletin s odporúčaním, bez diverzifikácie;}
    \item{Skupina B -- informačný bulletin s odporúčaním, metóda proporčnej diverzifikácie;}
    \item{Skupina C -- informačný bulletin s odporúčaním, metóda tématického vzorkovania;}
\end{my_enumerate}

Dôležitým predpokladom pre úspešnosť online experimentu bude získať dostatočnú reprezentatívnu vzorku používateľov
ochotných byť súčasťou experimentu. V~prípade, že by sa nám nepodarilo osloviť dostatočný počet používateľov, plánujeme
vykonať kontrolovaný offline experiment na vzorke archívnych dát.


\section{Metriky hodnotenia výsledkov}

Pre overovanie výsledkov experimentov budeme používať nasledovné metriky:

\textbf{Precision@N}\\
Presnosť (angl.~\emph{Precision}), alebo tiež \textit{pozitívna predikčná hodnota} je metrika reprezentujúca pomer relevantných
dokumentov z celkového zoznamu. Štandardne sa presnosť počíta ako pomer z celého zoznamu dokumentov, no v oblasti
odporúčania a vyhľadávania informácií je často vhodnejšou odvodená metrika \textit{Precision@N}, ktorá určuje, aká časť
z prvých N dokumentov v~zozname odporúčaní je pre používateľa relevantná.
$$\mbox{Precision@N}=\frac{|\{\mbox{relevantne otazky v top-N}\}\cap\{\mbox{top-N odporucenych otazok}\}|}{|\{\mbox{top-N odporucenych otazok}\}|}$$

\textbf{nDCG}\\
\textit{Normalized Discounted Cumulative Gain} je metrika kvality ohodnocovania často používaná na meranie efektívnosti
odporúčania. DCG meria užitočnosť dokumentov na základe ich pozície vo výslednom zozname. Užitočnosť dokumentov sa akumuluje
od konca zoznamu, pričom najvyššiu užitočnosť majú dokumenty na začiatku zoznamu~\cite{Jrvelin2002}.

nDCG položky na pozícii $p$ v zozname odporúčaní je definované ako:

$$\mathrm{nDCG_{p}} = \frac{DCG_{p}}{IDCG_{p}}$$
$$\mathrm{DCG_{p}} = \sum_{i=1}^{p} \frac{ 2^{rel_{i}} - 1 }{ \log_{2}(i+1)}$$
$$\mathrm{IDCG_{p}} = \sum_{i=1}^{|REL|} \frac{ 2^{rel_{i}} - 1 }{ \log_{2}(i+1)}$$
\begin{adjustwidth}{1cm}{1cm}
$rel_i$ -- relevancia $i$-tej položky v zozname odporúčaní.\\
$|REL|$ -- zoznam $p$ relevantných položiek zo zoznamu odporúčaní usporiadaných podľa relevancie.
\end{adjustwidth}

\textbf{CTR}\\
Miera preklikov (angl.~\emph{Click-through Rate}) je metrika často využívaná v spojitosti s informačnými bulletinmi.
Táto metrika vyjadruje počet úspešných kliknutí na odkaz v informačnom bulletine.

Okrem CTR plánujeme v súvislosti s interakciou používateľa s informačným bulletinom merať aj počet impresií,
teda zobrazení informačného bulletinu, ako aj počet konverzií, teda podiel prípadov, kedy kliknutie na niektorú z otázok
v informačnom bulletine viedlo k aktivite používateľa na tejto otázke -- či už označenie za obľúbenú,
odpovedanie alebo pridanie komentáru. Ďalej tiež plánujeme merať počet odhlásení z informačného bulletinu.
