%!TEX root = main.tex
\afterpage{\blankpage}
\newpage
\chapter{Úvod}

Informačné bulletiny (angl.~\emph{newsletters}) sú stále jednou z najčastejšie používaných foriem informovania používateľov
webových portálov o novinkách, akciách alebo zaujímavom obsahu na webe. Používatelia radi využívajú informačné bulletiny
svojich obľúbených webových portálov, pretože predstavujú jednoduchý a~prehľadný spôsob prezentácie obsahu, ktorý je navyše
doručený pohodlne priamo do používateľovej e-mailovej schránky.

Informačné bulletiny zastávajú ešte väčšiu rolu v rámci online komunít, ktoré produkujú veľké množstvo používateľmi
vytváraného obsahu. Medzi populárne druhy takýchto online komunít patria aj systémy pre odpovedanie na otázky
(angl.~\emph{Community Question Answering systems -- CQA}). Veľké množstvo obsahu, ktoré v~týchto systémoch vzniká,
si vyžaduje nasadzovanie personalizačných techník za~účelom poskytnutia relevantného obsahu používateľom.

Výskum v oblasti CQA systémov sa v súčasnosti skôr orientuje na skúmanie správania používateľov, kladenia otázok a odpovedania.
Problematike informačných bulletinov v doméne CQA systémov zatiaľ nebola venovaná dostatočná pozornosť a to napriek tomu,
že existujúce informačné bulletiny nespĺňajú očakávania komunity a sú pre ne charakterisické viaceré problémy, napr.
akým spôsobom ich vhodne personalizovať a ako zároveň zabezpečiť diverzitu ich obsahu.

Cieľom našej práce je analyzovať súčasný stav výskumu odporúčania otázok v~doméne CQA systémov a navrhnúť riešenie pre
tvorbu personalizovaných informačných bulletinov v systémoch pre odpovedanie na otázky. V našej práci sa~zameriavame
predovšetkým na skúmanie vplyvu zavedenia diverzity a podpory aktuálnosti na úspešnosť personalizovaného odporúčania
otázok vo forme informačných bulletinov.

Nasledujúca kapitola sa venuje informačným bulletinom -- čo sú to informačné bulletiny, aké je ich využitie v rôznych
systémoch a aké problémy sú s nimi spájané. Ďalšia kapitola rozoberá problematiku systémov pre odpovedanie na otázky,
opisuje ich rozdelenie a problémy a ďalej sa venuje doméne odporúčacích systémov a odporúčania v CQA systémoch.
Nasleduje kapitola opisujúca problematiku diverzity a aktuálnosti v odporúčaní.

Piata kapitola obsahuje samotný návrh riešenia personalizovaného informačného bulletinu v CQA systéme. Nasledujú kapitoly
opisujúce implementáciu navrhnutých metód a ich overenie v online nekontrolovanom experimente. Záver práce tvorí vyhodnotenie
a prílohy.
