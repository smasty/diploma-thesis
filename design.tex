%!TEX root = main.tex
\newpage
\chapter{Návrh riešenia personalizovaného informačného bulletinu v CQA systéme}

Cieľom našej práce je navrhnúť a overiť metódu zostavovania personalizovaných informačných bulletinov v CQA systémoch
so zameraním na podporu diverzity a aktuálnosti odporúčaného obsahu.

\section{Dáta v doméne CQA systémov}

Experimenty s personalizáciou informačných bulletinov budeme realizovať na platforme Stack Exchange, ktorá patrí medzi
najpopulárnejšie CQA systémy súčasnosti a tvorí ju viac ako 160 samostatných komunít zameraných na rôzne oblasti.

Experimenty plánujeme vykonávať nad dátami z komunity \textit{Software Engineering}\footnote{\url{http://softwareengineering.stackexchange.com}},
nakoľko je táto komunita zameraná na doménu, v ktorej máme hlbšie vedomosti a teda vieme lepšie posúdiť správnosť
odporúčania v tejto doméne. Navrhnuté riešenie však nebude špecifické pre túto komunitu, našim cieľom je naopak vytvoriť
metódu, ktorú bude možné nasadiť v rámci celej platformy Stack Exchange, rovnako ako na iných CQA systémoch s podobnou
štruktúrou.

Platforma Stack Exchange pravidelne zverejňuje kompletné archívne dáta zo všetkých komunít vo forme XML výstupov reflektujúcich
entity systému. Tieto dáta plánujeme využiť v prvotnej fáze experimentov na vytvorenie základných modelov.
Pre následné zdokonaľovanie modelov, ako aj zabezpečenie aktuálnosti použitých dát budeme využívať verejné API poskytované
platformou Stack Exchange.

\subsection{Charakteristika dát}

\subsubsection{Archívne dáta}
Dáta v dátovom archíve platformy Stack Exchange\footnote{\url{http://archive.org/details/stackexchange}} sú pravidelne
aktualizované a obsahujú kompletné používateľmi vytvorené anonymizované dáta zo všetkých komunít platformy.
Všetky tieto dáta sú verejne dostupné pod licenciou \emph{Creative Commons Attribution-ShareAlike 3.0 Unported}.

Štruktúra dát v archívoch je nasledovná:

\begin{my_itemize}
  \item{\textit{Badges.xml} -- Obsahuje ID používateľov, názvy odznakov a čas, kedy používateľ daný odznak získal.}
  \item{\textit{Comments.xml} -- Obsahuje všetky komentáre spolu s informáciou o ich autoroch a príspevkoch, ku ktorým sa viažu.}
  \item{\textit{Posts.xml} -- Obsahuje informácie o všetkých príspevkoch (otázkach a odpovediach) a k nim prislúchajúce značky, ako aj aktuálne znenie príspevku}
  \item{\textit{PostHistory.xml} -- Obsahuje históriu zmien jednotlivých príspevkov, ako napr. zmenu názvu, štítkov, označenie otázky za zodpovedanú a pod.}
  \item{\textit{PostLinks.xml} -- Obsahuje informácie o prepojeniach medzi príspevkami, konkrétne o duplikátoch a príbuzných príspevkoch.}
  \item{\textit{Users.xml} -- Obsahuje verejné údajé všetkých používateľov, ako sú meno, reputácia, webová stránka, počet hlasov a iné.}
  \item{\textit{Votes.xml} -- Obsahuje anonymizované informácie o hlasoch príspevkov.}
\end{my_itemize}


\subsubsection{Stack Exchange API}

API platformy Stack Exchange\footnote{\url{http://api.stackexchange.com}} poskytuje prístup ku všetkým verejným dátam
platformy v reálnom čase. API podporuje dvojúrovňový prístup k dátam.

\textbf{Obmedzenia požiadaviek}\\
Menšie aplikácie môžu využiť základnú úroveň, ktorá na používanie nevyžaduje registráciu a autentifikáciu,
no jej prístup podlieha obmedzovaniu v miere prístupu a môže z jednej IP adresy denne vykonať iba 10 000 požiadaviek na API.

V prípade, že aplikácia využívajúca API vykoná autentifikáciu používateľa, je limit 10 000 požiadaviek denne špecifický
pre každého používateľa zvlášť.

Bez ohľadu na úroveň prístupu je tiež uplatňované limitovanie na 30 požiadaviek za sekundu z jednej IP adresy, ako aj
dynamické limitovanie, ktoré spočíva v tom, že každá odpoveď API môže obsahovať parameter indikujúci počet sekúnd, koľko
má aplikácia počkať pred vykonaním ďalšej požiadavky. Tento limit musí dodržiavať každá aplikácia.

\textbf{Vlastné filtre}\\
Stack Exchange API umožňuje flexibilne špecifikovať jednotlivé atribúty, ktoré majú byť vrátené v odpovedi na požiadavku.
Táto podpora je implementovaná prostredníctvom filtrov, ktoré môže aplikácia využívajúca API vytvoriť. Každý filter môže
obsahovať (1) zoznam atribútov, ktoré majú byť prítomné, (2) atribúty, ktoré nemajú byť súčasťou odpovede a (3) základný
filter, od ktorého je daný filter odvodený.\\
Použitie filtrov umožňuje aplikáciám pokladať efektívnejšie požiadavky, ktoré vrátia iba všetky aplikáciou požadované
atribúty a nič navyše.

\textbf{Autentifikácia}\\
Autentifikácia používateľov pri použití API je implementovaná prostredníctvom otvoreného štandardu
OAuth 2.0\footnote{\url{http://oauth.net}}. Pre využívanie autentifikovaných požiadaviek v API je nutné zaregistrovať
aplikáciu v centrálnom zozname všetkých aplikácií využívajúcich API platformy Stack Exchange --
\textit{Stack Apps}\footnote{\url{http://stackapps.com}}.

\subsubsection{Rozsah dát}

Komunita \textit{Software Engineering}, ktorú budeme využívať v experimentoch pri návrhu riešenia má nasledovný rozsah:

\begin{my_itemize}
    \item{Otázky - cca 45 tisíc}
    \item{Odpovede - cca 138 tisíc}
    \item{Používatelia - cca 221 tisíc}
    \item{Pridelené odznaky - cca 405 tisíc}
    \item{Komentáre - cca 397 tisíc}
    \item{Značky - cca 1.6 tisíc}
    \item{Miera zodpovedanosti otázok - cca 94\%}
\end{my_itemize}


\section{Návrh metódy personalizovaného odporúčania}

Hypoteza

Ako to bude fungovat

Riesenie cold startu

Ake crty chcem pouzit

Ake metriky chcem pouzit

\section{Návrh metódy diverzifikácie odporúčaní}

Hypoteza

Ako to bude fungovat - chcem porovnat dva pristupy.

Ake crty chcem pouzit

Ake metriky chcem pouzit

\section{Návrh overenia metód}

online nekontrolovany experiment, A/B testovanie.

zalozny plan - offline kontrolovany experiment.
