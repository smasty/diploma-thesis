%!TEX root = main.tex
\afterpage{\blankpage}
\newpage
\chapter{Návrh riešenia personalizovaného informačného bulletinu v CQA systéme}

Cieľom našej práce je navrhnúť, zrealizovať a overiť metódu zostavovania personalizovaných informačných bulletinov v CQA systémoch
so zameraním sa na podporu diverzity a~aktuálnosti obsahu odporúčaného v informačnom bulletine.

Naše riešenie je navrhnuté pre použitie vrámci platformy Stack Exchange, ktorá patrí medzi najpopulárnejšie CQA systémy
v súčasnosti a tvorí ju viac ako 160 komunít zameraných na rôzne oblasti.

Informačný bulletin bude tvoriť viacero sekcií, ktoré vyžadujú rôzne spôsoby odporúčania. Konkrétne ide o odporúčanie
nových otázok a odporúčanie vyriešených otázok, pričom v~prvom prípade je dôležité pozerať sa na expertízu používateľa
a v druhom prípade na oblasti jeho záujmu.

\textbf{Hypotéza 1}\\
\textit{Použitím personalizovaného odporúčania otázok v informačnom bulletine CQA systému zvýšime relevanciu obsahu informačného
bulletinu, čo sa prejaví zvýšenou mierou jeho používania medzi používateľmi CQA systému.}

\textbf{Hypotéza 2}\\
\textit{Výberom odporúčaných otázok zo širšieho okruhu záujmu používateľa a zohľadnením ich aktuálnosti predídeme výskytu problému
filtračnej bubliny, čím dosiahneme vyššiu mieru záujmu používateľa a jeho aktivity.}

\textbf{Prehľad navrhovanej metódy}\\
Na začiatok uvádzame stručný súhrn navrhnutej metódy zostavovania personalizovaných informačných bulletinov.
Celkový pohľad na metódu ilustruje obrázok~\ref{fig:overview}, Jednotlivé časti metódy sú podrobne opísané v nasledujúcich kapitolách.
\begin{my_itemize}
\item{
    Pre zostavenie personalizovaných informačných bulletinov využijeme odporúčaciu metódu filtrovania na základe obsahu
    (kap.~\ref{rec:content}), keďže je v prípade malej používateľskej aktivity efektívnejšia ako kolaboratívne filtrovanie
    (viď kap.~\ref{cold-start}).
}
\item{
    Diverzifikáciu odporúčaného obsahu budeme zabezpečovať na úrovni tématického pre-filteru pred samotným procesom
    odporúčania -- pre každú nezávislú tému sa bude vytvárať samostatný zoznam odporúčaní.
    Pri návrhu tejto metódy sme vychádzali z~\cite{Szpektor2013}.
}
\item{Výsledný zoznam odporúčaného obsahu vznikne spojením jednotlivých zoznamov z~kroku diverzifikácie.}
\item{Pri odporúčaní sa okrem diverznosti a relevancie obsahu bude uvažovať aj jeho aktuálnosť.}
\item{
    Navrhnutá metóda bude reflektovať implicitnú aj explicitnú spätnú väzbu používateľa vo forme interakcie
    s informačným bulletinom a profil používateľa sa bude na jej základe aktualizovať.
}
\end{my_itemize}

\begin{figure}[H]\begin{center}
\includegraphics[scale=0.32]{method-diagram}
\caption{Schéma metódy zostavovania personalizovaných informačných bulletinov.\label{fig:overview}}\end{center}
\end{figure}

\section{Návrh metódy personalizovaného odporúčania}


Pri vytváraní personalizovaného informačného bulletinu sa budeme zameriavať na odporúčanie relevantných otázok jednotlivým
používateľom CQA systému prostredníctvom aplikovania metódy filtrovania na základe obsahu (kapitola~\ref{rec:content}).
Pre účely filtrovania na základe obsahu je potrebné definovať a zostaviť profily reprezentujúce jednak otázky, a tiež
používateľov CQA systému.

\subsection{Profil otázok}

Profil otázky sa bude skladať z troch nezávislých modelov, ktoré sa na samotnú otázku pozerajú z rôznych perspektív.

\textbf{1. Kategorický model otázok}\\
Tento model reprezentuje otázku na najvyššej úrovni ako prislúchajúcu do určitých kategórií. Kategórie otázok sú vrámci
platformy Stack Exchange reprezentované ako značky (angl.~\textit{tags}). Každá otázka môže obsahovať viacero značiek.

Samotný kategorický model otázky bude reprezentovaný ako vektor v $n$-rozmernom priestore, kde každý rozmer $n_i$
predstavuje príslušnosť otázky k danej značke. Nakoľko značky netvoria hierarchickú štruktúru, bude vektor v jednotlivých
rozmeroch nadobúdať iba hodnoty $0$~alebo~$1$.

\textbf{2. Tématický model otázok}\\
Tématický model využíva metódu latentnej Dirichletovej alokácie (angl.~\emph{Latent Dirichlet Allocation - LDA})~\cite{blei2003latent}
na určenie latentnej témy, ktorej sa daná otázka venuje.

LDA vektor tohto modelu bude reprezentovať distribúciu otázky vrámci jednotlivých latentných tém.
Samotné LDA témy sa budú odvodzovať z nadpisu a textu otázky.
Pre optimalizáciu modelu a zanedbanie tém s veľmi nízkou distribúciou sa do úvahy budú brať len latentné témy tvoriace
75\% z celkovej distribúcie, teda tretí kvartil. Jednotlivé hodnoty tém budú následne normalizované, aby tvorili 100\%.

\begin{figure}[H]\begin{center}
\includegraphics[scale=0.3]{lda-plate}
\caption{Platňová notácia LDA modelu.\protect\footnote}
\end{center}\end{figure}
\footnotetext{Prevzaté 10.12.2017 z \url{https://en.wikipedia.org/wiki/Latent_Dirichlet_allocation}}

\begin{adjustwidth}{1cm}{1cm}
\textbf{Nastavenie LDA}\\
\label{design:lda-setup}
Pre určenie vhodného počtu LDA tém ($n$) využijeme metódu hierarchických Dirichletových procesov~\cite{Teh2006}.
Pre trénovanie LDA modelu bude využitý online variačný Bayesov algoritmus. Parametre modelu budú nastavené nasledovne:\\
$$\alpha = \frac{1}{n}; \kappa = 0.7; \tau_0 = 10; \eta = \frac{1}{n}$$

$M$ -- Celkový počet dokumentov. \\
$N$ -- Celkový počet slov vo všetkých dokumentoch. \\
$Z$ -- $N$-rozmerný vektor identít tém všetkých slov vo všetkých dokumentoch. \\
$W$ -- $N$-rozmerný vektor identít všetkých slov vo všetkých dokumentoch. \\
$n$ -- Celkový počet tém. \\
$\alpha$   -- Úvodná váha tém v dokumente. $1/n$ zabezpečí v úvode rovnomernú distribúciu.\\
$\beta$    -- Úvodná váha slov v téme. $1/n$ zabezpečí v úvode rovnomernú distribúciu.\\
$\theta_d$ -- Distribúcia tém v dokumente $d$. \\
$\kappa$   -- Úpadok učenia; parameter kontrolujúci mieru učenia v online metóde učenia. \\
$\tau_0$   -- Posun učenia; znižuje váhu počiatočným iteráciám v online metóde učenia.

Nakoľko jednotlivé komunity vrámci platformy Stack Exchange sú pomerne úzko zamerané, predpokladáme, že bude dostačujúce
natrénovať LDA model na vzorke archívnych dát a nebude potrebné postupné dotrénovanie modelu. Napriek tomu sme zvolili
online variačný Bayesov algoritmus, keďže je pri veľkom množstve dát efektívnejší ako dávková varianta tohto algoritmu.
\end{adjustwidth}

\textbf{3. Lexikálny model otázok}\\
Lexikálny model otázky využíva TF-IDF vektor reprezentujúci zastúpenie jednotlivých výrazov v texte otázky. Rovnako ako
LDA vektor bude zostavený z nadpisu a textu samotnej otázky. Pred výpočtom bude text lematizovaný a budú z neho odstránené
stop slová.

$$\textrm{tf-idf}(t, d) = \textrm{tf}(t, d) \times \log\frac{n_d}{1+\textrm{df}(d, t)}$$

\subsection{Profil používateľov}

Pre každého používateľa budeme uvažovať dva nezávislé podprofily -- jeden bude profilovať záujem používateľa o určité témy a otázky,
druhý bude profilovať jeho expertízu v určitej oblasti. Jeden bude použitý
pri odporúčaní otázok, ktoré by používateľa mohli zaujímať, druhý pri odporúčaní otázok, ktoré by mohol vedieť zodpovedať.
Oba podprofily budú z pohľadu svojej štruktúry presne rovnaké.

Takýto prístup k rozdeleniu profilu používateľa podľa záujmu a expertízy je vo svojej podstate unikátny, nakoľko ho využíva
iba práca~\cite{Xu2012}. V ostatný prácach autori buď modelujú len jeden z týchto aspektov, alebo ich vôbec nerozlišujú~\cite{Srba2016}.

Profil používateľa bude zostavený na základe jeho aktivity a bude sa analogicky k profilu otázok skladať z troch vektorov:

\begin{my_enumerate}
\item{
  Prvý vektor bude reprezentovať aktivitu používateľa naprieč značkami -- každý rozmer bude predstavovať jednu značku,
  v ktorej má používateľ aktivitu. Hodnoty v jednotlivých rozmeroch budú predstavovať podiel aktivity v danej značke voči
  celkovému množstvu aktivity používateľa.
}
\item{
  Druhý vektor bude analogicky k prvému reprezentovať aktivitu používateľa naprieč LDA témami, v ktorých má používateľ aktivitu.
}
\item{
  Tretí vektor bude reprezentovať zastúpenie jednotlivých výrazov v~textoch otázok, v~ktorých má používateľ aktivitu.
}
\end{my_enumerate}

\textbf{Záujmový podprofil}\\
Podprofil predstavujúci záujem používateľa bude zostavený zo všetkých otázok, ktoré používateľ položil, alebo ktoré označil
za obľúbené. Každá takáto otázka bude do podprofilu prispievať rovnakou váhou.

\textbf{Expertízny podprofil}\\
Podprofil modelujúci expertízu používateľa sa bude skladať z otázok, ktoré používateľ označil za obľúbené alebo
na ktoré používateľ odpovedal, pričom ich dopad na podprofil expertízy bude závislý od skóre jeho odpovede.
Nezáporné skóre bude prispievať do podprofilu pozitívne -- bude to teda signalizovať
fakt, že používateľ danej téme rozumie. Odpoveď so záporným skóre bude naopak signalizovať, že používateľ danej téme nerozumie,
čo bude reflektované aj v jeho expertíznom podprofile. Odpovede označené za akceptované budú do podprofilu prispievať
s~koeficientom $1.75$, čo je priemerný počet odpovedí na otázky s akceptovanou odpoveďou.

\textbf{Komentáre}\\
Okrem pokladania otázok, odpovedania alebo označenia za obľúbené budú uvažované aj používateľove komentáre. Keďže však
zo samotného faktu že používateľ niečo okomentoval nemožno jednoznačne určiť, či táto aktivita predstavuje jeho záujem
alebo expertízu, budú otázky, ktoré používateľ okomentoval prispievať do celého profilu -- záujmového aj expertízneho, avšak
s~koeficientom $0.3$. Tento koeficient vychádza z faktu, že vrámci platformy Stack Exchange pripadajú na jednu
komentovanú otázku alebo odpoveď priemerne tri komentáre.

\textbf{Spätná väzba}\\
Do oboch podprofilov budú využitím rovnakého princípu prispievať aj implicitná a explicitná spätná väzba používateľa --
teda jeho kliknutie na konkrétnu otázku v informačnom bulletine, alebo jej explicitné kladné alebo záporné ohodnotenie.
Implicitná spätna väzba bude prispievať s koeficientom $0.3$, explicitná s koeficientom $\pm1$.

\textbf{Vzorce pre zostavenie záujmového a expertízneho podprofilu používateľa}\\
$$\mathbf{UP_{interest}} = \sum_{q \in Qa, Qf}T_q + \sum_{q \in Qc, Qif}0.3 \times T_q + \sum_{q \in Qef}\pm1 \times T_q$$
$$\mathbf{UP_{expertise}} = \sum_{q \in Qw, Qf}T_q + \sum_{q \in Qac}1.75 \times T_q + \sum_{q \in Qc, Qif}0.3 \times T_q + \sum_{q \in Qef}\pm1 \times T_q$$
\begin{adjustwidth}{1cm}{1cm}
$T_q$ -- profil otázky $q$.\\
$Qa$ -- množina položených otázok\\
$Qf$ -- množina otázok označených za obľubené\\
$Qc$ -- množina komentovaných otázok\\
$Qw$ -- množina zodpovedaných otázok\\
$Qac$ -- množina zodpovedaných otázok s označením akceptovanej odpovede\\
$Qif$ -- množina otázok s implicitnou spätnou väzbou\\
$Qef$ -- množina otázok s explicitnou spätnou väzbou\\
\end{adjustwidth}

\subsection{Výber odporúčaného obsahu}\label{design:rec-retrieval}

Pre účely zostavovania zoznamu odporúčaných otázok použijeme prístup analogický štandardným nástrojom pre vyhľadávanie
informácií (angl.~\emph{Information Retrieval Engines}). V našom prípade budú dokumentmi samotné otázky a dopytom bude
príslušný profil používateľa. Pre ohodnocovanie podobnosti profilov bude použitý skalárny súčin vektorov reprezentujúcich
profily otázky a používateľa.

\textbf{Skalárny súčin (angl.~\textit{dot product}) vektorov $\mathbf{a}$ a $\mathbf{b}$.}\\
$$\mathbf{a}\cdot\mathbf{b}=\sum_{i=1}^n a_ib_i=a_1b_1+a_2b_2+\cdots+a_nb_n$$

\subsection{Riešenie problému studeného štartu}

Pre eliminovanie problému studeného štartu (kapitola~\ref{cold-start}) z pohľadu prvotného odporúčania otázok využijeme
offline natrénovanie našich metód na archívnych dátach platformy Stack~Exchange.

V prípade nových používateľov, ktorí v systéme nemajú žiadnu, alebo iba nedostatočnú aktivitu, sa zo začiatku namiesto profilu
konkrétneho používateľa použije komunitný profil, ktorý je zostavený analogicky k profilu používateľa, no pozostáva
z celkovej aktivity v systéme.


\subsection{Aktuálnosť odporúčaní}
\label{design:freshness}
Pri zostavovaní odporúčaní do informačného bulletinu sa budú uvažovať len otázky, ktoré pochádzajú
z obdobia od posledného zostavenia informačného bulletinu.

Aby profil používateľa reflektoval meniace sa záujmy používateľa v čase, zavedieme pri zostavovaní profilu z používateľovej
aktivity exponenciálny \textit{faktor úpadku} (angl.~\textit{decay factor}):
$$y_{n+t} = y_n (1 - d)^t$$

\begin{adjustwidth}{1cm}{1cm}
$y_n$ -- aktuálna váha danej veličiny v modeli používateľa\\
$y_{n+t}$ -- váha danej veličiny po aplikovaní faktoru úpadku\\
$t$ -- čas od pregenerovania modelu používateľa -- počet pregenerovaní modelu.\\
$d$ -- percentuálny pokles danej veličiny; vypočíta sa ako podiel množstva používateľovej aktivity za čas $t$ a jeho celkového množstva aktivity.
\end{adjustwidth}

\section{Návrh metódy diverzifikácie odporúčaní}

Diverzifikácia obsahu personalizovaného informačného bulletinu sa bude vykonávať ešte pred samotným zostavovaním
zoznamu odporúčaní formou pre-filteru. Diverzifikácia bude spočívať vo výbere \textit{značiek} a \textit{tém}, pričom
následne v kroku zostavovania zoznamov odporúčaní sa bude pre každý zoznam uvažovať len obsah prislúchajúci do danej
značky alebo témy. Okrem výberu značiek a tém bude diverzifikácia aplikovaná aj pri následnom výbere položiek z jednotlivých
zoznamov do výsledného zoznamu odporúčaní.

Tento prístup k diverzifikácii sme zvolili okrem jeho prirodzenosti aj z dôvodu jeho veľmi dobrej škálovateľnosti,
nakoľko alternatívny prístup postavený na tvorbe odporúčaní nad všetkým obsahom daného CQA systému a až následnej
diverzifikácií by v praxi nebol realizovateľný. Ako príklad môžeme uviesť systém Stack Overflow, kde denne pribudne
cca 8000-9000 nových otázok, čo by v prípade generovania týždenného informačného bulletinu znamenalo vypočítať podobnosť
až cca 60000 profilov otázok s profilom každého používateľa.

Cieľom našej práce je vyhodnotiť dopad diverzifikácie odporúčaní na personalizované informačné bulletiny CQA systémov.
Za týmto účelom sme navrhli metódu tématického vzorkovania \textit{(angl. Thematic sampling)}.

\subsection{Metóda tématického vzorkovania}
\label{sec:thematic-sampling}
Pri návrhu tejto metódy sme sa inšpirovali rovnomennou metódou z~\cite{Szpektor2013}, ktorú sme prispôsobili
podmienkam špecifickým pre našu prácu. Proces tématického vzorkovania ilustruje schéma na obrázku~\ref{fig:tematic-sampling}.
\begin{my_enumerate}
\item{Pre každého používateľa vyberieme náhodne z jeho profilu $\left\lceil\frac{n}{2}\right\rceil$~značiek
    a~$\left\lfloor\frac{n}{2}\right\rfloor$~tém, kde $n$ je počet položiek vo výslednom zozname. Pri výbere sa obmedzíme na značky a témy,
    ktoré spadajú do štvrtého kvartilu, najmenej však vyberáme z piatich. Potlačíme tak značky a témy, ktoré majú pre používateľa nízku relevanciu.}
\item{
    Prostredníctvom vyššie opísanej metódy (kapitola~\ref{design:rec-retrieval}) sa pre každú takúto značku a tému
    následne zostaví zoznam $n$ odporúčaní, pričom sa budú uvažovať len otázky prislúchajúce tejto téme alebo značke.}
\item{
    Následne náhodne vyberáme vzorky z jednotlivých zoznamov do výsledného zoznamu odporúčaní, pričom pravdepodobnosť
    výberu otázky z určitého zoznamu je proporčná k relevancii tohto zoznamu pre daného používateľa.}
\item{
    Konkrétne otázky do výsledného zoznamu odporúčaní z jednotlivých zoznamov sa budú vyberať z~$m$ najrelevantnejších otázok daného zoznamu,
    pričom $m$ je relatívna relevancia tohto zoznamu.}
\end{my_enumerate}

\begin{figure}[H]\begin{center}
\includegraphics[scale=0.27]{diversity-diagram}
\caption{Schéma metódy diverzifikácie odporúčaní tématickým vzorkovaním.\label{fig:tematic-sampling}}\end{center}
\end{figure}
