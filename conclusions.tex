%!TEX root = main.tex
\afterpage{\blankpage}
\newpage
\chapter{Zhodnotenie}

Cieľom našej práce bolo analyzovať súčasný stav výskumu odporúčania otázok v~doméne CQA systémov a navrhnúť riešenie pre
tvorbu personalizovaných informačných bulletinov v systémoch pre odpovedanie na otázky. Zamerali sme sa predovšetkým na
skúmanie vplyvu zavedenia diverzity a podpory aktuálnosti na úspešnosť personalizovaného odporúčania.

Navrhli sme metódu personalizovaného odporúčania obsahu v doméne CQA systémov vo forme informačných bulletinov so zameraním
sa na podporu aktuálnosti odporúčaní, ktorú sme doplnili o diverzifikačnú metódu tématického vzorkovania.

Pre účely overenia navrhnutých metód sme pripravili experimentálnu infraštruktúru StackLetter, ktorá umožňuje jendoduché
vytváranie a rozosielanie personalizovaných informačných bulletinov pre používateľov platformy Stack Exchange.
Navrhnuté metódy sme overili prostredníctvom online nekontrolovaného experimentu na používateľoch komunity Stack Overflow.

Výsledky experimentu ukázali, že personalizácia obsahu informačných bulletinov prostredníctvom odporúčacích metód
a s dôrazom na aktuálnosť odporúčaného obsahu má pozitívny vplyv na záujem používateľov o informačné bulletiny, ako
aj na mieru ich celkovej aktivity v systéme. Experiment nepotvrdil hypotézu, že používatelia budú pozitívne vnímať
aj diverzifikáciu prezentovaných odporúčaní. Naopak sa ukázalo, že používatelia nevnímajú prítomnosť filtračnej bubliny
ako problém, ale ako žiadaný jav.
V online experimente sa tiež prejavili zakorenené predsudky voči informačným bulletinom ako takým, čo spôsobilo náročné
získavanie používateľov do realizovaného online nekontrolovaného experimentu.

Celkovo hodnotíme prácu ako úspešnú, nakoľko ukázala, aký vplyv má personalizácia obsahu informačných bulletinov v CQA systémoch,
jeho aktuálnosť a tiež diverzita na odoberateľov a ich spokojnosť, záujem a aktivitu.


\textbf{Možný výskum v budúcnosti}\\
Zaujímavou oblasťou potenciálneho výskumu v doméne informačných bulletinov CQA systémov, ale aj informačných bulletinov
všeobecne, sa javí práve skúmanie spomínaných zakorenených predsudkov voči nim, a vnímanie informačných bulletinov ako spamu.
Môže tiež byť zaujímavé skúmať aj ďalšie metódy \textit{oživenia} klasických personalizačných metód,
aj keď realizovaný experiment naznačuje, že používatelia sú zvyknutí práve na štandardnú personalizáciu.
Okrem samotného zamerania sa na personalizáciu v informačných bulletinoch CQA systémov je možné skúmať nové možnosti využitia
informačných bulletinov v týchto systémoch.
