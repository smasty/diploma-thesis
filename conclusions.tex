%!TEX root = main.tex
\newpage

\chapter{Zhodnotenie}

\textbf{TODO}\\
-- zhodnotenie vysledkov - jedna hypoteza vysla, druha nie.\\
-- online experiment bol unikatny, tazko realizovatelny, napriek tomu priniesol zaujimave vysledky\\
-- to ze ludia chcu filter bubble je mozno este zaujimavejsie ako keby to nechceli\\
-- diverzitu prirodzene skor vnimaju ako negativum, su zvyknuti ze kazdy im chce co najviac personalizovat, hlavne ak
to tvrdi nadpis newslettera.\\
-- historicka stigma okolo nwsletterov, ludia ich moc nechcu odoberat, casto vnimaju ako len dalsi spam, a ak sa prihlasia
tak po case velmi rychlo prestanu byt aktivni.

-- \textbf{Future work}\\
-- explorovat \textit{stigmu} okolo newsletterov, ine metody "ozivenia" standardnej personalizacie,
aj ked sa momentalne zda ze pouzivatelia chcu hlavne cisto personalizovane veci.\\
-- ine vyuzitie newsletterov v CQA, aj mimo personalizacie.



% V súčasnosti je systém StackLetter nasadení v testovacej prevádzke. Zatiaľ používateľom umožňuje nechať si odosielať
% informačné bulletiny pre komunity \textit{Stack Overflow} a \textit{Software Engineering} buď raz denne, alebo raz týždenne.

% \begin{table}[h]
% \centering
% \caption{Používanosť systému StackLetter}
% \begin{tabular}{|m{10cm}|m{1.5cm}|}
% \hline
% Počet odoberateľov informačného bulletinu & 24 \\ \hline
% Počet odoslaných informačných bulletinovu & 493 \\ \hline
% Počet používateľmi otvorených informačných bulletinov & 119\footnotemark \\ \hline
% Počet otvorených položiek informačných bulletinov & 145 \\ \hline
% Počet explicitných hodnotení informačných bulletinov & 114 \\ \hline
% \end{tabular}
% \end{table}

% \footnotetext{Počet otvorení môže byť skreslený, nakoľko viaceré klienty blokujú zobrazenie obrázkov použitých na vyhodnotenie otvorenia e-mailu.}

% Obsah informačného bulletinu je zatiaľ personalizovaný len jednoduchou metódou na základe značiek, v ktorých má používateľ aktivitu.
% Návrh metód odporúčania s~podporou diverzity a~aktuálnosti je v tejto fáze práce už finalizovaný.

% V nasledujúcom semestri plánujeme implementovať metódy odporúčania s~podporou diverzity a~aktuálnosti a ďalej pokračovať
% v online nekontrolovanom experimente. V prípade vyhodnotenia tohto experimentu ako neúspešného máme v pláne vykonať
% online kontrolovaný experiment. Podrobnejší plán práce v ďalšom semestri obsahuje príloha~\ref{apx:dp3plan}.
